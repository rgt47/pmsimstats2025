\documentclass[11pt]{article}
\usepackage[margin=1in]{geometry}
\usepackage{amsmath}
\usepackage{amssymb}
\usepackage{amsthm}
\usepackage{listings}
\usepackage{xcolor}
\usepackage{hyperref}
\usepackage{booktabs}
\usepackage{enumitem}
\usepackage{graphicx}
\usepackage{tikz}

\theoremstyle{definition}
\newtheorem{definition}{Definition}
\newtheorem{example}{Example}
\newtheorem{intuition}{Intuition}

% R code styling
\lstset{
  language=R,
  basicstyle=\ttfamily\small,
  keywordstyle=\color{blue},
  commentstyle=\color{gray},
  stringstyle=\color{red},
  breaklines=true,
  frame=single,
  numbers=left,
  numberstyle=\tiny\color{gray}
}

\title{Understanding Correlation Parameters in N-of-1 Trial Simulations:\\
A Practical Guide}
\author{Technical Documentation}
\date{November 12, 2025}

\begin{document}

\maketitle

\begin{abstract}
This document provides a comprehensive guide to understanding and selecting the six correlation parameters that define the covariance structure in N-of-1 trial simulations. We explain what each parameter represents, provide intuitive interpretations, and give practical guidance for choosing appropriate values. This guide complements the theoretical treatment in \texttt{correlation\_structure\_design.pdf}.
\end{abstract}

\tableofcontents
\newpage

\section{The Six Correlation Parameters}

\subsection{Overview}

The correlation structure for N-of-1 trial simulations is completely specified by \textbf{six parameters}. These parameters control how measurements relate to each other across time and across the three response components (time-variant, pharmacologic/expectancy, and biological response).

\subsection{Parameter Definitions}

\begin{table}[h]
\centering
\caption{The Six Correlation Parameters}
\begin{tabular}{@{}lllll@{}}
\toprule
\textbf{Parameter} & \textbf{Name} & \textbf{Meaning} & \textbf{Hendrickson} & \textbf{PDF Rec.} \\
\midrule
\texttt{c.tv} & TV autocorrelation & Corr(tv$_t$, tv$_s$) for $t \neq s$ & 0.8 & 0.65 \\
\texttt{c.pb} & PB autocorrelation & Corr(pb$_t$, pb$_s$) for $t \neq s$ & 0.8 & 0.65 \\
\texttt{c.br} & BR autocorrelation & Corr(br$_t$, br$_s$) for $t \neq s$ & 0.8 & 0.65 \\
\texttt{c.cf1t} & Same-time cross & Corr(comp1$_t$, comp2$_t$) & 0.2 & 0.18 \\
\texttt{c.cfct} & Diff-time cross & Corr(comp1$_t$, comp2$_s$) for $t \neq s$ & 0.1 & 0.09 \\
\texttt{c.bm} & Biomarker & Corr(biomarker, br$_t$) & 0--0.6 & 0--0.6 \\
\bottomrule
\end{tabular}
\end{table}

\noindent where:
\begin{itemize}
\item \textbf{tv} = time-variant factor (natural disease progression)
\item \textbf{pb} = pharmacologic/biomarker factor (expectancy/placebo effect)
\item \textbf{br} = biological response factor (true drug effect)
\item \textbf{comp1, comp2} = any pair of different components (tv, pb, or br)
\end{itemize}

\subsection{Visual Structure}

For a design with 3 timepoints (t1, t2, t3), the correlation matrix has this structure:

\begin{equation}
\mathbf{R} = \begin{bmatrix}
\mathbf{I}_2 & \mathbf{0} & \mathbf{0} & \mathbf{0} & \mathbf{0} \\
\mathbf{0} & \mathbf{R}_{\text{tv}} & \mathbf{C}_{12} & \mathbf{C}_{13} \\
\mathbf{0} & \mathbf{C}_{21} & \mathbf{R}_{\text{pb}} & \mathbf{C}_{23} \\
\mathbf{0} & \mathbf{C}_{31} & \mathbf{C}_{32} & \mathbf{R}_{\text{br}}
\end{bmatrix}
\end{equation}

where:
\begin{itemize}
\item $\mathbf{I}_2$ represents biomarker and baseline (uncorrelated by construction)
\item $\mathbf{R}_{\text{tv}}$ has \texttt{c.tv} in all off-diagonal positions
\item $\mathbf{C}_{ij}$ has \texttt{c.cf1t} on diagonal, \texttt{c.cfct} off-diagonal
\item Biomarker row/column has \texttt{c.bm} for BR positions only
\end{itemize}

\subsection{Expanded Example}

For 3 timepoints, the full correlation matrix looks like:

\begin{small}
\[
\begin{array}{c|ccccccccccc}
 & \text{bm} & \text{BL} & \text{tv}_1 & \text{tv}_2 & \text{tv}_3 & \text{pb}_1 & \text{pb}_2 & \text{pb}_3 & \text{br}_1 & \text{br}_2 & \text{br}_3 \\
\hline
\text{bm} & 1 & 0 & 0 & 0 & 0 & 0 & 0 & 0 & c.bm & c.bm & c.bm \\
\text{BL} & 0 & 1 & 0 & 0 & 0 & 0 & 0 & 0 & 0 & 0 & 0 \\
\text{tv}_1 & 0 & 0 & 1 & c.tv & c.tv & c.cf1t & c.cfct & c.cfct & c.cf1t & c.cfct & c.cfct \\
\text{tv}_2 & 0 & 0 & c.tv & 1 & c.tv & c.cfct & c.cf1t & c.cfct & c.cfct & c.cf1t & c.cfct \\
\text{tv}_3 & 0 & 0 & c.tv & c.tv & 1 & c.cfct & c.cfct & c.cf1t & c.cfct & c.cfct & c.cf1t \\
\text{pb}_1 & 0 & 0 & c.cf1t & c.cfct & c.cfct & 1 & c.pb & c.pb & c.cf1t & c.cfct & c.cfct \\
\text{pb}_2 & 0 & 0 & c.cfct & c.cf1t & c.cfct & c.pb & 1 & c.pb & c.cfct & c.cf1t & c.cfct \\
\text{pb}_3 & 0 & 0 & c.cfct & c.cfct & c.cf1t & c.pb & c.pb & 1 & c.cfct & c.cfct & c.cf1t \\
\text{br}_1 & c.bm & 0 & c.cf1t & c.cfct & c.cfct & c.cf1t & c.cfct & c.cfct & 1 & c.br & c.br \\
\text{br}_2 & c.bm & 0 & c.cfct & c.cf1t & c.cfct & c.cfct & c.cf1t & c.cfct & c.br & 1 & c.br \\
\text{br}_3 & c.bm & 0 & c.cfct & c.cfct & c.cf1t & c.cfct & c.cfct & c.cf1t & c.br & c.br & 1 \\
\end{array}
\]
\end{small}

\subsection{What Each Parameter Controls}

\subsubsection{Within-Component Autocorrelations (Parameters 1--3)}

\begin{lstlisting}[language=R]
# c.tv controls ALL these correlations:
Corr(tv_1, tv_2) = 0.8
Corr(tv_1, tv_3) = 0.8
Corr(tv_2, tv_3) = 0.8
# Same pattern for c.pb and c.br
\end{lstlisting}

\textbf{Key Point}: Same correlation regardless of time lag (not AR(1)-style decay).

\subsubsection{Cross-Component, Same Time (Parameter 4)}

\begin{lstlisting}[language=R]
# c.cf1t controls same-time cross-component correlations:
Corr(tv_1, pb_1) = 0.2  # time 1
Corr(tv_2, pb_2) = 0.2  # time 2
Corr(tv_3, pb_3) = 0.2  # time 3
Corr(tv_1, br_1) = 0.2
Corr(pb_1, br_1) = 0.2
# etc. for all same-time pairs
\end{lstlisting}

\subsubsection{Cross-Component, Different Times (Parameter 5)}

\begin{lstlisting}[language=R]
# c.cfct controls different-time cross-component correlations:
Corr(tv_1, pb_2) = 0.1  # tv at time 1, pb at time 2
Corr(tv_1, br_3) = 0.1  # tv at time 1, br at time 3
Corr(pb_1, br_2) = 0.1  # pb at time 1, br at time 2
# etc. for all different-time pairs
\end{lstlisting}

\subsubsection{Biomarker-Response (Parameter 6)}

\begin{lstlisting}[language=R]
# c.bm controls ONLY biomarker-BR correlations:
Corr(biomarker, br_1) = 0.3
Corr(biomarker, br_2) = 0.3
Corr(biomarker, br_3) = 0.3

# NOT correlated with tv or pb:
Corr(biomarker, tv_1) = 0
Corr(biomarker, pb_1) = 0
\end{lstlisting}

\subsection{Critical Hierarchy Constraint}

For the correlation matrix to be positive definite, you \textbf{must maintain}:

\begin{equation}
\boxed{c.cfct < c.cf1t < \min(c.tv, c.pb, c.br)}
\end{equation}

\begin{example}[Valid Hierarchy]
\begin{lstlisting}
c.tv = 0.8, c.pb = 0.8, c.br = 0.8    # autocorrelations
c.cf1t = 0.2                           # 0.2 < 0.8 (checkmark)
c.cfct = 0.1                           # 0.1 < 0.2 (checkmark)
\end{lstlisting}
\end{example}

\begin{example}[Invalid Hierarchy - FAILS!]
\begin{lstlisting}
c.tv = 0.5                             # autocorrelation
c.cf1t = 0.1                           # same-time cross
c.cfct = 0.4                           # 0.4 > 0.1 VIOLATION!
\end{lstlisting}

This violates temporal coherence: measurements at \textbf{different times} cannot be more correlated than measurements at the \textbf{same time}.
\end{example}

\section{Recommended Parameter Values}

\subsection{Three Standard Configurations}

\subsubsection{Hendrickson Exact (for direct comparison)}

\begin{lstlisting}
model_params <- list(
  c.tv = 0.8,
  c.pb = 0.8,
  c.br = 0.8,
  c.cf1t = 0.2,
  c.cfct = 0.1,
  c.bm = 0.3      # or vary: 0, 0.3, 0.6
)
\end{lstlisting}

\textbf{Properties}:
\begin{itemize}
\item Empirically validated (Hendrickson et al., 2020)
\item Represents strong individual-level stability
\item Works for $T = 8$ timepoints (hybrid design)
\item Best for direct comparison with published results
\end{itemize}

\subsubsection{Intermediate (PDF Section 9.1 recommendation)}

\begin{lstlisting}
model_params <- list(
  c.tv = 0.65,
  c.pb = 0.65,
  c.br = 0.65,
  c.cf1t = 0.18,  # 0.28 * 0.65
  c.cfct = 0.09,  # 0.50 * 0.18
  c.bm = 0.3      # or vary
)
\end{lstlisting}

\textbf{Properties}:
\begin{itemize}
\item Guaranteed PD for $T \leq 20$ with $c.bm \leq 0.6$
\item Maintains proper correlation hierarchy
\item Interpretable: moderate individual-level stability
\item Allows moderate parameter variation
\end{itemize}

\subsubsection{Conservative (always works)}

\begin{lstlisting}
model_params <- list(
  c.tv = 0.5,
  c.pb = 0.5,
  c.br = 0.5,
  c.cf1t = 0.15,  # 0.30 * 0.5
  c.cfct = 0.08,  # 0.50 * 0.15
  c.bm = 0.3
)
\end{lstlisting}

\textbf{Properties}:
\begin{itemize}
\item Guaranteed PD for $T \leq 30$
\item Very stable across parameter variations
\item Lower power for interaction detection
\item Represents moderate temporal persistence
\end{itemize}

\subsection{Recommended Ratios (PDF Guideline 2)}

To ensure positive definiteness, use these ratios:

\begin{align}
c.cf1t &\approx 0.2 \text{ to } 0.4 \times c_{\text{autocorr}} \\
c.cfct &\approx 0.5 \text{ to } 0.7 \times c.cf1t
\end{align}

\section{Intuitive Understanding of Within-Component Autocorrelations}

\subsection{What Does Autocorrelation Represent?}

Within-component autocorrelation (e.g., \texttt{c.tv = 0.8}) measures how consistent an individual's \textbf{deviation from the mean} is across time.

\begin{intuition}[High Autocorrelation]
\textbf{High autocorrelation (0.8)} means:

``If someone is above average at time 1, they'll probably be above average at times 2, 3, 4...''

\textbf{Implication}: People have stable ``types'' or consistent individual characteristics.
\end{intuition}

\begin{intuition}[Low Autocorrelation]
\textbf{Low autocorrelation (0.3)} means:

``If someone is above average at time 1, they might be anywhere at time 2---above, below, who knows.''

\textbf{Implication}: People's measurements fluctuate randomly; no stable individual characteristics.
\end{intuition}

\subsection{Component-Specific Interpretations}

\subsubsection{Time-Variant Factor (c.tv)}

\textbf{What it represents}: Natural disease trajectory, independent of treatment

\paragraph{High c.tv = 0.8 (``Stable disease trajectory'')}

\begin{itemize}
\item \textbf{Interpretation}: People have consistent symptom patterns over time
\item \textbf{Example}: Person A always has severe symptoms, Person B always has mild symptoms, regardless of treatment
\item \textbf{Real-world analogy}: Chronic pain with stable severity
\item \textbf{Data pattern}: If you plot symptoms over time, each person's line stays in their ``lane'' (high/medium/low)
\end{itemize}

\begin{verbatim}
Symptoms over time with c.tv = 0.8:

High severity |  *---*---*---*---*  (Person A: consistently high)
              |
Medium        |      o---o---o---o  (Person B: consistently medium)
              |
Low severity  |          .---.---.  (Person C: consistently low)
              +---------------------
              t1   t2  t3  t4  t5
\end{verbatim}

\paragraph{Low c.tv = 0.3 (``Fluctuating disease'')}

\begin{itemize}
\item \textbf{Interpretation}: Disease severity bounces around unpredictably
\item \textbf{Example}: Person A might be severe at t1, mild at t2, severe again at t3
\item \textbf{Real-world analogy}: Episodic conditions (migraines, flare-ups)
\item \textbf{Data pattern}: Spaghetti plot---lines cross each other constantly
\end{itemize}

\begin{verbatim}
Symptoms over time with c.tv = 0.3:

High severity |  *       o---.
              |   \     /     \
Medium        |    o---*       *
              |   /     \     /
Low severity  |  .       .---o
              +---------------------
              t1   t2  t3  t4  t5
              (lots of crossing!)
\end{verbatim}

\subsubsection{Pharmacologic/Expectancy Factor (c.pb)}

\textbf{What it represents}: Placebo/expectancy response

\paragraph{High c.pb = 0.8 (``Consistent placebo responders'')}

\begin{itemize}
\item \textbf{Interpretation}: Some people always have strong placebo response, others never do
\item \textbf{Example}: Person A always gets 5-point boost from placebo, Person B always gets 1-point boost
\item \textbf{Real-world analogy}: Trait-like ``suggestibility''---stable individual characteristic
\item \textbf{Implication}: Placebo response is a person-specific trait
\end{itemize}

\paragraph{Low c.pb = 0.3 (``Variable placebo response'')}

\begin{itemize}
\item \textbf{Interpretation}: Placebo effectiveness varies within same person across trials
\item \textbf{Example}: Person A gets 5-point boost on Monday, 1-point boost on Friday
\item \textbf{Real-world analogy}: Context-dependent placebo effects (mood, stress, etc.)
\item \textbf{Implication}: Placebo response depends on state, not trait
\end{itemize}

\subsubsection{Biological Response Factor (c.br)}

\textbf{What it represents}: True drug effect (beyond placebo)

\paragraph{High c.br = 0.8 (``Stable drug responders'')}

\begin{itemize}
\item \textbf{Interpretation}: If drug works well for you at t1, it'll work well at t2, t3...
\item \textbf{Example}: Person A always gets 10-point improvement from drug, Person B always gets 3-point improvement
\item \textbf{Real-world analogy}: Pharmacogenetics---your genetics determine drug response consistently
\item \textbf{Implication}: ``Responders'' vs ``non-responders'' is a stable classification
\end{itemize}

\paragraph{Low c.br = 0.3 (``Inconsistent drug response'')}

\begin{itemize}
\item \textbf{Interpretation}: Same person responds differently at different times
\item \textbf{Example}: Person A: 10-point improvement week 1, 2-point improvement week 3
\item \textbf{Real-world analogy}: Tolerance, receptor saturation, or environmental interactions
\item \textbf{Implication}: No stable ``responder'' types---response varies within person
\end{itemize}

\subsection{Mathematical Intuition}

\subsubsection{What the Correlation Measures}

For person $i$, let their TV component at time $t$ be: $\text{tv}_{it}$

The \textbf{autocorrelation} is:
\begin{equation}
c.tv = \text{Corr}(\text{tv}_{i1}, \text{tv}_{i2}) = \text{Corr}(\text{tv}_{i1}, \text{tv}_{i3}) = \cdots
\end{equation}

This measures: \textbf{Do individual differences persist over time?}

\subsubsection{Decomposing the Variance}

Each measurement can be decomposed as:
\begin{equation}
\text{tv}_{it} = \underbrace{\mu_t}_{\text{overall mean}} + \underbrace{\alpha_i}_{\substack{\text{stable} \\ \text{person effect}}} + \underbrace{\epsilon_{it}}_{\substack{\text{random} \\ \text{fluctuation}}}
\end{equation}

\textbf{High autocorrelation} $\Rightarrow$ Large $\alpha_i$, small $\epsilon_{it}$
\begin{itemize}
\item Variance mostly \textit{between} people (stable differences)
\item Little variance \textit{within} people over time
\end{itemize}

\textbf{Low autocorrelation} $\Rightarrow$ Small $\alpha_i$, large $\epsilon_{it}$
\begin{itemize}
\item Variance mostly \textit{within} people over time
\item Little stable individual difference
\end{itemize}

\subsubsection{Variance Partition}

If $c.tv = 0.8$:
\begin{equation}
\text{Total Variance} = \underbrace{80\%}_{\text{between-person}} + \underbrace{20\%}_{\text{within-person}}
\end{equation}

If $c.tv = 0.3$:
\begin{equation}
\text{Total Variance} = \underbrace{30\%}_{\text{between-person}} + \underbrace{70\%}_{\text{within-person}}
\end{equation}

This is closely related to the \textbf{Intraclass Correlation Coefficient (ICC)}:
\begin{equation}
\text{ICC} = \frac{\sigma^2_{\text{between}}}{\sigma^2_{\text{between}} + \sigma^2_{\text{within}}} \approx c.\text{tv}
\end{equation}

\section{Effect on Statistical Power}

\subsection{High Autocorrelation (c.tv = 0.8)}

\subsubsection{Advantages}
\begin{itemize}
\item[\checkmark] \textbf{Better power to detect between-person effects} (biomarker $\times$ treatment interaction)
\item[\checkmark] Stable individual differences are easy to detect
\item[\checkmark] Biomarker can predict who responds well (if \texttt{c.bm} is also high)
\end{itemize}

\subsubsection{Disadvantages}
\begin{itemize}
\item[$\times$] \textbf{Harder to see within-person treatment effects} (less room for change)
\item[$\times$] If someone starts high, they tend to stay high even with treatment
\end{itemize}

\subsubsection{Best For}
Detecting \textbf{moderators} (who benefits?) rather than main effects

\subsection{Low Autocorrelation (c.tv = 0.3)}

\subsubsection{Advantages}
\begin{itemize}
\item[\checkmark] \textbf{More room for treatment effects} to show through
\item[\checkmark] Measurements are more independent $\Rightarrow$ more effective sample size
\item[\checkmark] Can see within-person changes more clearly
\end{itemize}

\subsubsection{Disadvantages}
\begin{itemize}
\item[$\times$] \textbf{Noisy baseline measurements} make biomarker prediction harder
\item[$\times$] Hard to identify stable ``responder'' types
\item[$\times$] Lower power for interaction detection
\end{itemize}

\subsubsection{Best For}
Detecting \textbf{main effects} (does treatment work overall?)

\section{Why Hendrickson Chose High Autocorrelations}

\subsection{Hendrickson's Choice: c.tv = c.pb = c.br = 0.8}

\subsubsection{Theoretical Justification}

\begin{enumerate}
\item \textbf{N-of-1 trials assume stable individual characteristics}
\begin{itemize}
\item The whole point is: ``Find what works for THIS person''
\item Requires that ``this person'' has stable trait-like responses
\end{itemize}

\item \textbf{Interaction detection requires between-person variance}
\begin{itemize}
\item Testing biomarker $\times$ treatment needs people to differ consistently
\item If everyone's bouncing around randomly, you can't predict who responds
\end{itemize}

\item \textbf{Realistic for chronic conditions}
\begin{itemize}
\item Chronic pain, depression: fairly stable severity within person
\item Treatment response often shows stable individual differences
\end{itemize}
\end{enumerate}

\subsubsection{Practical Impact}

\begin{lstlisting}
c.tv = 0.8 means:
- ICC (Intraclass Correlation) ~ 0.8
- 80% of variance is between-person
- Only 20% is random within-person fluctuation
\end{lstlisting}

This creates \textbf{strong individual signatures} that persist across time.

\subsection{Visual Comparison}

\subsubsection{High Autocorrelation (c.br = 0.8): ``Responder Types''}

\begin{verbatim}
Drug Response by Person (each line = 1 person):

Strong      |  =========== Person A (consistent strong responder)
Responder   |  ===========
            |
Moderate    |      -------  Person B (consistent moderate)
            |      -------
            |
Weak        |          ...  Person C (consistent weak)
Responder   |          ...
            +------------------------
            t1  t2  t3  t4  t5  t6

-> You can classify people into types
-> Biomarker can predict these stable types
-> Good for precision medicine
\end{verbatim}

\subsubsection{Low Autocorrelation (c.br = 0.3): ``Variable Response''}

\begin{verbatim}
Drug Response by Person:

Strong      |  =     .   -   =
            |    -   =   .     .
Moderate    |  .   =   =   .   -
            |    .   -   =   -   =
Weak        |  -   -   .   -   .
            +------------------------
            t1  t2  t3  t4  t5  t6

-> Same person bounces between strong/weak
-> Hard to classify people into types
-> Biomarker can't predict (response is unstable)
-> Bad for precision medicine, but realistic for some drugs
\end{verbatim}

\section{Connection to Study Design}

\subsection{Crossover Designs (N-of-1)}

\textbf{High autocorrelation is NECESSARY}:
\begin{itemize}
\item You're comparing drug vs placebo \textbf{within the same person}
\item Requires that person's \textbf{baseline tendency} is stable
\item Otherwise you can't tell if change is due to treatment vs random fluctuation
\end{itemize}

\begin{example}[Crossover Data with Different Autocorrelations]
Person A's symptoms across alternating drug/placebo periods:

\begin{center}
\begin{tabular}{lcccc}
\toprule
& \textbf{Drug} & \textbf{Placebo} & \textbf{Drug} & \textbf{Placebo} \\
Week: & 1 & 2 & 3 & 4 \\
\midrule
\textbf{Low c} (0.3): & 7 & 5 & 9 & 3 \\
\textbf{High c} (0.8): & 7 & 9 & 7 & 9 \\
\bottomrule
\end{tabular}
\end{center}

\textbf{Analysis}:
\begin{itemize}
\item With \textbf{low autocorrelation}: Can't tell drug effect from noise (baseline bouncing 3--9)
\item With \textbf{high autocorrelation}: Clear drug effect---person's baseline is stable around 8, drug consistently causes $-2$ point reduction
\end{itemize}
\end{example}

\subsection{Practical Implications for Your Simulation}

\subsubsection{What c.tv = 0.8 means for your results}

\begin{enumerate}
\item \textbf{Biomarker interactions will be easier to detect}
\begin{itemize}
\item Stable individual differences $\Rightarrow$ biomarker can predict them
\item $c.bm = 0.6$ will have strong effect on power
\end{itemize}

\item \textbf{Designs with more participants will do better than more timepoints}
\begin{itemize}
\item Between-person variance is where the action is
\item 70 participants better than 35 with more measurements per person
\end{itemize}

\item \textbf{Carryover effects are more problematic}
\begin{itemize}
\item Person's baseline is stable, so carryover ``sticks around''
\item Washout periods are more important
\end{itemize}
\end{enumerate}

\subsubsection{What c.tv = 0.3 would mean}

\begin{enumerate}
\item \textbf{Harder to detect interactions}
\begin{itemize}
\item Biomarker can't predict unstable responses
\item $c.bm$ effects would be diluted
\end{itemize}

\item \textbf{More timepoints per person helps}
\begin{itemize}
\item Within-person variance is where the action is
\item Repeated measures improve precision
\end{itemize}

\item \textbf{Carryover less problematic}
\begin{itemize}
\item Random fluctuation swamps carryover signal
\item But also swamps treatment signal!
\end{itemize}
\end{enumerate}

\section{Choosing Values for Your Simulation}

\subsection{Autocorrelation Guidelines}

\begin{table}[h]
\centering
\caption{Autocorrelation Value Interpretations}
\begin{tabular}{@{}llp{6cm}@{}}
\toprule
\textbf{Value} & \textbf{Interpretation} & \textbf{When Realistic} \\
\midrule
0.3--0.5 & Conservative & Moderate stability; episodic conditions \\
0.5--0.7 & Moderate & Most chronic conditions; typical temporal stability \\
0.7--0.9 & High & Very stable traits; pharmacogenetics; N-of-1 context \\
\bottomrule
\end{tabular}
\end{table}

\subsubsection{Conservative (c = 0.5)}
\begin{itemize}
\item \textbf{Interpretation}: ``Half the variance is stable individual differences, half is random''
\item \textbf{When realistic}: Moderate stability conditions
\item \textbf{Power implications}: Moderate power for interactions
\end{itemize}

\subsubsection{Moderate-High (c = 0.65)}
\begin{itemize}
\item \textbf{Interpretation}: ``2/3 stable, 1/3 random''
\item \textbf{When realistic}: Most chronic conditions
\item \textbf{Power implications}: Good power for interactions
\end{itemize}

\subsubsection{High (c = 0.8---Hendrickson)}
\begin{itemize}
\item \textbf{Interpretation}: ``4/5 stable, 1/5 random''
\item \textbf{When realistic}: Very stable traits, pharmacogenetics
\item \textbf{Power implications}: High power for interactions
\end{itemize}

\subsection{Summary Table}

\begin{table}[h]
\centering
\caption{Configuration Comparison}
\begin{tabular}{@{}llll@{}}
\toprule
\textbf{Aspect} & \textbf{Conservative} & \textbf{Intermediate} & \textbf{Hendrickson} \\
\midrule
Autocorrelations & 0.5 & 0.65 & 0.8 \\
Variance partition & 50-50 & 65-35 & 80-20 \\
Individual types & Moderate & Clear & Very clear \\
Interaction power & Moderate & Good & High \\
Main effect power & High & Moderate & Lower \\
Stability & Very high & High & Moderate \\
Use case & Exploration & Balanced & Comparison \\
\bottomrule
\end{tabular}
\end{table}

\section{Key Takeaways}

\subsection{The Six Parameters Control Everything}

These six parameters completely determine the correlation structure:
\begin{enumerate}
\item \texttt{c.tv, c.pb, c.br}: Within-component temporal stability
\item \texttt{c.cf1t}: Cross-component synchrony (same time)
\item \texttt{c.cfct}: Cross-component lag correlation (different times)
\item \texttt{c.bm}: Biomarker predictive value
\end{enumerate}

\subsection{Within-Component Autocorrelation is Critical}

\textbf{High autocorrelation (0.6--0.8)} answers:
\begin{quote}
``Are people consistently different from each other over time?''
\end{quote}

\textbf{Answer: YES}
\begin{itemize}
\item People have stable ``types''---responders vs non-responders
\item Essential for N-of-1 trials and precision medicine
\item Required for biomarker prediction to work
\end{itemize}

\subsection{Always Maintain the Hierarchy}

The single most important constraint:
\begin{equation}
\boxed{c.cfct < c.cf1t < \min(c.tv, c.pb, c.br)}
\end{equation}

Violating this hierarchy \textbf{guarantees} a non-positive-definite matrix.

\subsection{Hendrickson's Choice is Well-Justified}

For N-of-1 trials focused on detecting biomarker $\times$ treatment interactions:
\begin{itemize}
\item \textbf{High autocorrelations (0.8)} are theoretically appropriate
\item Reflects stable individual characteristics
\item Enables precision medicine approach
\item Empirically validated in published simulations
\end{itemize}

\subsection{Practical Recommendations}

\begin{enumerate}
\item \textbf{For Hendrickson comparison}: Use exact values (all 0.8, 0.2, 0.1)
\item \textbf{For general use}: Use intermediate values (0.65, 0.18, 0.09)
\item \textbf{For sensitivity}: Test conservative to high range
\item \textbf{Always}: Validate positive definiteness before running simulations
\end{enumerate}

\section{References}

\begin{enumerate}
\item Hendrickson, E., Hatfield, L. A., \& Hodges, J. S. (2020). N-of-1 trials with multiple randomization structures: Design, power, and carryover effects.

\item See companion document: \texttt{correlation\_structure\_design.pdf} for theoretical treatment and troubleshooting

\item See companion document: \texttt{correlation\_structure\_discussion.md} for additional context and recommendations
\end{enumerate}

\end{document}
