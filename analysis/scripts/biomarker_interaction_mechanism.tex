\documentclass[12pt,oneside]{article}

\usepackage[margin=1in]{geometry}
\usepackage{amsmath}
\usepackage{amssymb}
\usepackage{booktabs}
\usepackage{xcolor}
\usepackage{fancyhdr}
\usepackage{listings}
\usepackage{xfrac}
\usepackage{mathtools}
\usepackage{hyperref}

% R code formatting
\lstset{
  language=R,
  basicstyle=\ttfamily\small,
  keywordstyle=\color{blue},
  commentstyle=\color{gray},
  stringstyle=\color{red},
  breaklines=true,
  showstringspaces=false,
  columns=flexible,
  backgroundcolor=\color{gray!10}
}

\pagestyle{fancy}
\fancyhf{}
\rhead{Biomarker-Treatment Interaction}
\lhead{Clustered N-of-1 Simulation}
\cfoot{\thepage}

\title{\textbf{Biomarker-Treatment Interaction Mechanism}}
\author{Clustered N-of-1 Trial Simulation}
\date{December 2025}

\begin{document}

\maketitle

\begin{abstract}
This document provides a comprehensive mathematical and computational explanation of how the clustered N-of-1 simulation generates biomarker-treatment interactions through two complementary mechanisms: (1) covariance structure via biomarker-response correlation, and (2) mean-level modulation via treatment effect scaling. These mechanisms work synergistically to create realistic pharmacogenomics scenarios and extend the Hendrickson et al.\ (2020) framework to explicitly study biomarker-driven heterogeneity in treatment response.
\end{abstract}

\tableofcontents
\newpage

% ============================================================================
\section{Overview}
% ============================================================================

The clustered simulation generates biomarker-treatment interactions through two distinct mechanisms:

\begin{enumerate}
  \item \textbf{Covariance structure}: Biomarker and responses are correlated in the sampling distribution via the cross-covariance matrix $\mathbf{\Sigma}_{12}$
  \item \textbf{Mean-level modulation}: The drug response rate itself is scaled by the observed biomarker level through multiplicative modulation
\end{enumerate}

These mechanisms work together to produce realistic heterogeneous treatment responses where individual biomarker status fundamentally shapes the magnitude of treatment benefit.

% ============================================================================
\section{Terminology and Parameter Definitions}
% ============================================================================

To ensure clarity throughout, we establish consistent naming conventions:

\begin{table}[h]
\centering
\small
\begin{tabular}{llll}
\toprule
\textbf{Concept} & \textbf{Code Variable} & \textbf{param\_grid Name} & \textbf{Controls} \\
\midrule
Baseline response value & \texttt{baseline} & — & Random intercept per participant \\
Biomarker value & \texttt{biomarker} & — & Individual characteristic (random) \\
BM-treatment interaction & \texttt{bm\_mod} & \texttt{biomarker\_moderation} & Effect modulation (\textbf{MEANS}) \\
BM-response correlation & \texttt{c.bm} & \texttt{biomarker\_correlation} & Covariance structure (\textbf{COVARIANCE}) \\
\bottomrule
\end{tabular}
\caption{Consistent terminology used throughout the simulation.}
\end{table}

\begin{description}
  \item[biomarker\_moderation ($\text{bm\_mod}$):] Controls the strength of biomarker modulation of the drug response rate. Affects \textbf{means only}.

  \item[biomarker\_correlation ($c.bm$):] Controls the correlation between biomarker and response variables in the covariance matrix. Affects \textbf{covariance structure only}.
\end{description}

% ============================================================================
\section{Two-Level Interaction Mechanism}
% ============================================================================

The interaction emerges from two complementary pathways:

\subsection{Level 1: Covariance Structure via Biomarker-Response Correlation}

The parameter $c.bm$ (biomarker\_correlation) creates correlation between the biomarker and responses through the cross-covariance matrix $\mathbf{\Sigma}_{12}$. This is implemented using conditional normal sampling:

\begin{lstlisting}
# Stage 1: Generate baseline values independently
stage1 <- mvrnorm(1, mu = c(biomarker_mean, baseline_mean), Sigma = Sigma_22)
biomarker <- stage1[1]
baseline <- stage1[2]

# Stage 2: Generate responses CONDITIONAL on biomarker
z <- c(biomarker - biomarker_mean, baseline - baseline_mean)
cond_mean <- as.vector(Sigma_12 %*% solve(Sigma_22) %*% z)
responses <- mvrnorm(1, mu = cond_mean, Sigma = Sigma_cond)
\end{lstlisting}

\textbf{Effect:} Observed biomarker values shift the response distributions upward or downward proportionally to their deviation from the mean. Higher $c.bm$ creates stronger shifts.

\subsection{Level 2: Treatment Effect Modulation via Biomarker-Treatment Interaction}

The parameter $\text{bm\_mod}$ (biomarker\_moderation) scales the drug response rate multiplicatively:

\begin{lstlisting}
bm_centered = (biomarker - biomarker_mean) / biomarker_sd
effective_BR_rate <- BR_rate * (1 + bm_mod * bm_centered)

# Example: BR_rate = 0.5, bm_mod = 0.45, biomarker is 1 SD above mean
# effective_BR_rate = 0.5 * (1 + 0.45 * 1) = 0.725

BR_mean = weeks_on_drug * effective_BR_rate
BR = BR_mean + br_random
response = baseline + BR + ER + TR
\end{lstlisting}

\textbf{Effect:} The drug effect slope itself varies by biomarker value through multiplicative scaling. High biomarker individuals experience stronger drug effects; low biomarker individuals experience weaker effects.

% ============================================================================
\section{Block-Partitioned Covariance Matrix Structure}
% ============================================================================

The full 26-dimensional covariance matrix is structured as a $2 \times 2$ block matrix:

\[
\boldsymbol{\Sigma} =
\begin{pmatrix}
\mathbf{\Sigma}_{11} & \mathbf{\Sigma}_{12} \\
\mathbf{\Sigma}_{12}^T & \mathbf{\Sigma}_{22}
\end{pmatrix}
\]

where:
\begin{align*}
\mathbf{\Sigma}_{11} &: 24 \times 24 \text{ (response covariances)} \\
\mathbf{\Sigma}_{12} &: 24 \times 2 \text{ (cross-covariance: responses and baseline)} \\
\mathbf{\Sigma}_{22} &: 2 \times 2 \text{ (baseline covariances)}
\end{align*}

\subsection{Sigma\_22: Covariance of Baseline Factors}

\[
\mathbf{\Sigma}_{22} =
\begin{pmatrix}
\sigma_{\text{BM}}^2 & \rho_{\text{BM,BL}} \sigma_{\text{BM}} \sigma_{\text{BL}} \\
\rho_{\text{BM,BL}} \sigma_{\text{BM}} \sigma_{\text{BL}} & \sigma_{\text{BL}}^2
\end{pmatrix}
\]

In the simulation:
\begin{lstlisting}
Sigma_22 <- matrix(c(
  biomarker_sd^2,  # = 4.0
  c.bm_baseline * biomarker_sd * between_subject_sd,  # = 0.3 * 2 * 2 = 1.2
  c.bm_baseline * biomarker_sd * between_subject_sd,
  between_subject_sd^2  # = 4.0
), 2, 2)
\end{lstlisting}

\subsection{Sigma\_11: Response Covariances}

The $24 \times 24$ response covariance matrix consists of three response factors at 8 timepoints:

\begin{align*}
\text{BR (Biological Response)} &: \text{8 timepoints, } c.br = 0.8 \text{ autocorrelation} \\
\text{ER (Expectancy Response)} &: \text{8 timepoints, } c.er = 0.8 \text{ autocorrelation} \\
\text{TR (Time-Variant Response)} &: \text{8 timepoints, } c.tr = 0.8 \text{ autocorrelation}
\end{align*}

Within-factor blocks use time-based AR(1) structure:

\[
\Sigma_{\text{BR}}[i,j] = \sigma^2 \cdot \rho^{|t_i - t_j|}
\]

Cross-factor blocks use compound symmetry with same-time ($c.cf1t = 0.2$) and different-time ($c.cfct = 0.1$) correlations.

\subsection{Sigma\_12: Cross-Covariance (Critical for Interaction)}

This $24 \times 2$ matrix encodes correlations between responses and baseline factors:

\begin{lstlisting}
Sigma_12[br_idx, 1] <- c.bm * within_subject_sd * biomarker_sd
Sigma_12[er_idx, 1] <- c.bm * 0.5 * within_subject_sd * biomarker_sd
Sigma_12[tr_idx, 1] <- c.bm * 0.5 * within_subject_sd * biomarker_sd
Sigma_12[br_idx, 2] <- c.baseline * within_subject_sd * between_subject_sd
Sigma_12[er_idx, 2] <- c.baseline * within_subject_sd * between_subject_sd
Sigma_12[tr_idx, 2] <- c.baseline * within_subject_sd * between_subject_sd
\end{lstlisting}

\textbf{Key observation:} Every element of $\mathbf{\Sigma}_{12}$ is scaled by $c.bm$ (biomarker\_correlation), making this matrix the locus of biomarker-response dependence.

% ============================================================================
\section{Concrete Example: How Biomarker Correlation Creates Response Shifts}
% ============================================================================

Assume:
\begin{align*}
c.bm &= 0.3 \text{ (biomarker\_correlation)} \\
\sigma_{\text{within}} &= 1.8 \text{ (within-subject SD)} \\
\sigma_{\text{BM}} &= 2.0 \text{ (biomarker SD)}
\end{align*}

The cross-covariance entry for biomarker $\to$ BR is:

\[
\mathbf{\Sigma}_{12}[\text{br\_idx}, 1] = 0.3 \times 1.8 \times 2.0 = 1.08
\]

\subsection{Scenario 1: High Biomarker (1 SD above mean)}

\begin{align*}
\text{biomarker} &= \mu_{\text{BM}} + 1 \sigma_{\text{BM}} = 5 + 2 = 7 \\
\mathbf{z} &= \begin{pmatrix} 7 - 5 \\ \text{baseline} - 10 \end{pmatrix} = \begin{pmatrix} 2 \\ 0 \end{pmatrix}
\end{align*}

Response shift via conditional distribution:

\[
\text{shift} = \mathbf{\Sigma}_{12} \mathbf{\Sigma}_{22}^{-1} \mathbf{z} \approx 1.08 \times 0.5 \times 2 = +1.08 \text{ units}
\]

\textbf{Result:} BR responses are generated approximately 1.08 units \textbf{higher} for high-biomarker individuals.

\subsection{Scenario 2: Low Biomarker (1 SD below mean)}

\begin{align*}
\text{biomarker} &= 5 - 2 = 3 \\
\mathbf{z} &= \begin{pmatrix} -2 \\ 0 \end{pmatrix}
\end{align*}

Response shift:

\[
\text{shift} = 1.08 \times 0.5 \times (-2) = -1.08 \text{ units}
\]

\textbf{Result:} BR responses are generated approximately 1.08 units \textbf{lower} for low-biomarker individuals.

% ============================================================================
\section{How Both Mechanisms Create Synergistic Interaction}
% ============================================================================

The two interaction mechanisms amplify each other:

\subsection{High Biomarker Individuals}

\begin{enumerate}
  \item \textbf{Covariance effect} ($c.bm$): Higher baseline response level
  \begin{equation}
    \text{response shift} = +1.08 \times \frac{\text{biomarker} - \mu_{\text{BM}}}{\sigma_{\text{BM}}}
  \end{equation}

  \item \textbf{Mean modulation effect} ($\text{bm\_mod}$): Stronger drug response
  \begin{equation}
    \text{effective\_BR\_rate} = 0.5 \times (1 + 0.45 \times 1) = 0.725
  \end{equation}

  \item \textbf{Combined result:} Higher baseline + stronger slope $\Rightarrow$ larger treatment effect $\Rightarrow$ easier to detect statistically
\end{enumerate}

\subsection{Low Biomarker Individuals}

\begin{enumerate}
  \item \textbf{Covariance effect:} Lower baseline response level
  \item \textbf{Mean modulation effect:} Weaker drug response
  \item \textbf{Combined result:} Lower baseline + weaker slope $\Rightarrow$ smaller treatment effect
\end{enumerate}

This synergy means that $c.bm$ (covariance correlation) and $\text{bm\_mod}$ (mean modulation) interact multiplicatively to determine effect sizes and statistical power.

% ============================================================================
\section{The Role of c.bm Parameter}
% ============================================================================

The parameter $c.bm$ controls the magnitude of covariance-based response shifts:

\subsection{Case: $c.bm = 0$ (No Correlation)}

\[
\mathbf{\Sigma}_{12}[\text{br\_idx}, 1] = 0 \times 1.8 \times 2.0 = 0
\]

\begin{itemize}
  \item Responses independent of biomarker in covariance structure
  \item No shift in conditional means
  \item Only mean-level interaction from $\text{bm\_mod}$
\end{itemize}

\subsection{Case: $c.bm = 0.3$ (Moderate Correlation)}

\[
\mathbf{\Sigma}_{12}[\text{br\_idx}, 1] = 0.3 \times 1.8 \times 2.0 = 1.08
\]

\begin{itemize}
  \item Responses shift by $\approx \pm 1$ unit per SD of biomarker
  \item Moderate covariance-based amplification
  \item Combined with $\text{bm\_mod}$ creates synergistic interaction
\end{itemize}

\subsection{Case: $c.bm = 0.6$ (High Correlation)}

\[
\mathbf{\Sigma}_{12}[\text{br\_idx}, 1] = 0.6 \times 1.8 \times 2.0 = 2.16
\]

\begin{itemize}
  \item Responses shift by $\approx \pm 2$ units per SD of biomarker
  \item Strong covariance-based amplification
  \item Powerful synergy with $\text{bm\_mod}$ produces largest effect sizes
\end{itemize}

% ============================================================================
\section{Implications for Statistical Power}
% ============================================================================

The covariance structure ($c.bm$) amplifies the mean-level interaction ($\text{bm\_mod}$):

\begin{table}[h]
\centering
\small
\begin{tabular}{cccc}
\toprule
\textbf{$c.bm$} & \textbf{bm\_mod} & \textbf{Effect} & \textbf{Power} \\
\midrule
0 & 0.5 & Only mean-level; modest & Low–Moderate \\
0.3 & 0.5 & Combined covariance + mean; larger & Moderate–High \\
0.6 & 0.5 & Strong covariance + mean; largest & High \\
\bottomrule
\end{tabular}
\caption{How $c.bm$ amplifies power to detect biomarker-treatment interaction.}
\end{table}

\subsection{Design of Parameter Grid}

The simulation contains:
\begin{align*}
\text{3 unique sigma structures} &: \text{(design} \times c.bm \text{ combinations)} \\
\text{36 parameter combinations} &: \text{including all } \text{bm\_mod} \text{ values}
\end{align*}

This design tests: \textit{Given a fixed covariance structure ($c.bm$), how does increasing biomarker\_moderation affect power to detect the interaction?}

The fact that only 3 sigma structures exist (despite 36 conditions) means:
\begin{itemize}
  \item Each unique ($\text{design}, c.bm$) pair generates the same correlation structure
  \item Different $\text{bm\_mod}$ values reuse the same sigma
  \item This isolates the effect of mean-level modulation from covariance effects
\end{itemize}

% ============================================================================
\section{Mathematical Formulation}
% ============================================================================

\subsection{Conditional Normal Distribution}

The responses conditional on baseline factors follow:

\begin{equation}
\mathbf{X}_1 \mid \mathbf{X}_2 = \mathbf{x}_2 \sim \mathcal{N}
\left(
\boldsymbol{\mu}_1 + \mathbf{\Sigma}_{12} \mathbf{\Sigma}_{22}^{-1}(\mathbf{x}_2 - \boldsymbol{\mu}_2),
\,
\boldsymbol{\Sigma}_{\text{cond}}
\right)
\end{equation}

where:
\begin{align*}
\mathbf{X}_1 &: \text{responses (24-dimensional)} \\
\mathbf{X}_2 &: \text{baseline factors (2-dimensional)} \\
\mathbf{\Sigma}_{12} \mathbf{\Sigma}_{22}^{-1}(\mathbf{x}_2 - \boldsymbol{\mu}_2) &: \text{shift dependent on observed biomarker} \\
\boldsymbol{\Sigma}_{\text{cond}} &= \mathbf{\Sigma}_{11} - \mathbf{\Sigma}_{12} \mathbf{\Sigma}_{22}^{-1} \mathbf{\Sigma}_{12}^T : \text{conditional covariance}
\end{align*}

\subsection{Treatment Effect Modulation}

The biological response rate is modulated as:

\begin{equation}
\text{BR}_{\text{rate}, i} = \text{BR}_{\text{base}} \times (1 + \text{bm\_mod} \times \tilde{x}_i)
\end{equation}

where:
\begin{align*}
\text{BR}_{\text{base}} &= 0.5 \quad \text{(fixed base rate)} \\
\text{bm\_mod} &\in \{0, 0.25, 0.35, 0.45, 0.55, 0.65\} \quad \text{(biomarker\_moderation)} \\
\tilde{x}_i &= \frac{x_i - \mu_x}{\sigma_x} \quad \text{(standardized biomarker for participant } i \text{)}
\end{align*}

The final response is:

\begin{equation}
Y_i = \text{baseline}_i + \text{BR}_i + \text{ER}_i + \text{TR}_i
\end{equation}

where $\text{BR}_i$, $\text{ER}_i$, $\text{TR}_i$ are generated from the conditional distribution with modulated mean for $\text{BR}_i$.

% ============================================================================
\section{Comparison to Hendrickson et al.\ (2020)}
% ============================================================================

Hendrickson et al.\ (2020) provided the foundational framework for N-of-1 trials with multiple randomization structures. The current simulation extends this work:

\begin{table}[h]
\centering
\small
\begin{tabular}{lll}
\toprule
\textbf{Aspect} & \textbf{Hendrickson} & \textbf{Current Simulation} \\
\midrule
Primary focus & Treatment detection & Biomarker $\times$ treatment \\
Biomarker role & Fixed (if present) & Systematically varied \\
BM-response correlation & Fixed/not studied & Varies: $\{0, 0.3\}$ \\
Interaction type & Not emphasized & Explicit interaction \\
Interaction mechanism & N/A & Multiplicative on effect rate \\
Covariance structure & Fixed Hendrickson params & Extended with BM correlation \\
Carryover modeling & Included & Included (extended) \\
\bottomrule
\end{tabular}
\caption{Comparison of Hendrickson et al.\ framework and current simulation.}
\end{table}

\subsection{Key Innovation}

The current simulation adds \textbf{explicit modeling of biomarker-driven heterogeneity} in treatment response through two mechanisms:

\begin{enumerate}
  \item \textbf{Structural:} Biomarker and responses are correlated in the sampling distribution
  \item \textbf{Functional:} Drug effect strength scales with biomarker value
\end{enumerate}

This enables power analysis for \textit{precision medicine} designs where treatment benefit is fundamentally linked to individual biomarker status—a core concept that Hendrickson's framework was not designed to study.

% ============================================================================
\section{Summary}
% ============================================================================

The biomarker-treatment interaction in the clustered N-of-1 simulation emerges from complementary mechanisms:

\begin{enumerate}
  \item \textbf{Covariance mechanism:} Parameter $c.bm$ creates correlation between biomarker and responses via $\mathbf{\Sigma}_{12}$, causing observed biomarker values to shift response distributions

  \item \textbf{Mean mechanism:} Parameter $\text{bm\_mod}$ scales drug effect rates multiplicatively, causing biomarker to modulate treatment slope
\end{enumerate}

Together, these mechanisms create:

\begin{itemize}
  \item \textbf{High biomarker individuals:} Higher baseline + stronger drug effect = larger detectable treatment benefit
  \item \textbf{Low biomarker individuals:} Lower baseline + weaker drug effect = smaller treatment benefit
  \item \textbf{Overall:} Heterogeneous treatment responses where biomarker status predicts both the baseline level and treatment response magnitude
\end{itemize}

This realistic pharmacogenomics scenario enables rigorous power analysis for precision medicine designs where treatment efficacy depends on individual biomarker characteristics.

% ============================================================================
\appendix
% ============================================================================

\section{Parameter Grid Structure}

\begin{lstlisting}
param_grid <- bind_rows(
  # OL+BDC power conditions
  expand_grid(
    design = "ol_bdc",
    biomarker_moderation = c(0.25, 0.35, 0.45, 0.55, 0.65),  # 5 values
    biomarker_correlation = c(0.3),                           # 1 value
    carryover = c(0, 0.5, 1)                                  # 3 values
  ),  # 15 combinations

  # OL+BDC Type I error
  expand_grid(
    design = "ol_bdc",
    biomarker_moderation = c(0),
    biomarker_correlation = c(0),
    carryover = c(0, 0.5, 1)
  ),  # 3 combinations

  # Hybrid design
  expand_grid(
    design = "hybrid",
    biomarker_moderation = c(0, 0.25, 0.35, 0.45, 0.55, 0.65),  # 6 values
    biomarker_correlation = c(0.3),                              # 1 value
    carryover = c(0, 0.5, 1)                                     # 3 values
  )   # 18 combinations
)
# Total: 36 combinations across 3 sigma structures
\end{lstlisting}

\section{Fixed Correlation Parameters}

All following parameters are held constant (from Hendrickson et al., 2020):

\begin{align*}
c.br &= 0.8 \quad \text{(BR autocorrelation)} \\
c.er &= 0.8 \quad \text{(ER autocorrelation)} \\
c.tr &= 0.8 \quad \text{(TR autocorrelation)} \\
c.cf1t &= 0.2 \quad \text{(same-time cross-correlation)} \\
c.cfct &= 0.1 \quad \text{(different-time cross-correlation)} \\
c.bm\_baseline &= 0.3 \quad \text{(biomarker-baseline correlation)} \\
c.baseline\_resp &= 0.4 \quad \text{(baseline-response correlation)}
\end{align*}

\end{document}
