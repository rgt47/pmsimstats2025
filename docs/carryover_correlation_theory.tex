\documentclass[11pt]{article}
\usepackage[margin=1in]{geometry}
\usepackage{amsmath}
\usepackage{amssymb}
\usepackage{listings}
\usepackage{xcolor}
\usepackage{hyperref}
\usepackage{booktabs}
\usepackage{enumitem}

% R code styling
\lstset{
  language=R,
  basicstyle=\ttfamily\small,
  keywordstyle=\color{blue},
  commentstyle=\color{gray},
  stringstyle=\color{red},
  breaklines=true,
  frame=single,
  numbers=left,
  numberstyle=\tiny\color{gray}
}

\title{Should Carryover Effects Modify Correlation Parameters in Clinical Trial Simulations?}
\author{Technical Analysis}
\date{\today}

\begin{document}

\maketitle

\begin{abstract}
This document examines whether carryover effects (residual treatment effects after discontinuation) should modify the correlation structure of multivariate normal distributions used in clinical trial simulations. We compare Hendrickson et al.'s approach (carryover affects means only) with an alternative approach (carryover affects both means and correlations), analyzing the theoretical and statistical implications of each.
\end{abstract}

\section{The Core Question}

Should carryover effects, which modify the mean structure of simulated data, also modify the correlation structure of the multivariate normal distribution from which we sample?

This question addresses a fundamental distinction between:
\begin{enumerate}
    \item \textbf{Mean structure} (expected values)
    \item \textbf{Covariance structure} (variability and correlations)
\end{enumerate}

\section{The Statistical Model}

The simulation generates data from a multivariate normal distribution:

\begin{equation}
\mathbf{Y} \sim \mathcal{N}(\boldsymbol{\mu}, \boldsymbol{\Sigma})
\end{equation}

where:
\begin{itemize}
    \item $\mathbf{Y}$ = vector of outcomes [biomarker, baseline, $\text{tv}_1, \text{tv}_2, \ldots, \text{pb}_1, \text{pb}_2, \ldots, \text{br}_1, \text{br}_2, \ldots$]
    \item $\boldsymbol{\mu}$ = means vector (includes carryover effects in Hendrickson's approach)
    \item $\boldsymbol{\Sigma}$ = covariance matrix (correlation structure $\times$ standard deviations)
\end{itemize}

\section{Argument 1: Carryover Should NOT Change Correlations}

\subsection{Position}

Carryover is a \textbf{mean effect}, not a \textbf{variance/correlation effect}.

\subsection{Reasoning}

\subsubsection{Carryover Represents Systematic Persistence}

Carryover represents systematic persistence of treatment effect:
\begin{itemize}
    \item If a drug effect at time $t$ was $+5$ points
    \item At time $t+1$ (off drug), we expect $+2.5$ points (with $t_{1/2} = 1$)
    \item This is a \textbf{shift in expected value}, not a change in variance
\end{itemize}

\subsubsection{Correlation Parameters Represent Residual Dependencies}

Correlation parameters represent residual dependencies after accounting for means:
\begin{itemize}
    \item $c_{\text{br}} = 0.8$ means: ``After accounting for treatment/carryover effects, an individual's deviation from expected response at time $t$ correlates 0.8 with their deviation at time $t+1$''
    \item This individual-level variability structure shouldn't change just because the population mean shifted
\end{itemize}

\subsubsection{Analogy to Linear Mixed Models}

Consider a standard linear mixed model:
\begin{equation}
Y_{ij} = \beta_0 + \beta_1 \cdot \text{Treatment} + \beta_2 \cdot \text{Carryover} + b_i + \varepsilon_{ij}
\end{equation}

where:
\begin{itemize}
    \item $\beta_2$ (carryover coefficient) affects means
    \item $\text{Var}(b_i)$ and $\text{Var}(\varepsilon_{ij})$ are variance components (unchanged by carryover)
    \item This is exactly what Hendrickson's approach implements
\end{itemize}

\subsubsection{Statistical Independence of Parameters}

Mean parameters and correlation parameters are orthogonal aspects of the distribution:
\begin{itemize}
    \item Mean parameters: $\{\text{max}, \text{disp}, \text{rate}, \text{carryover\_t1half}\}$
    \item Correlation parameters: $\{c_{\text{tv}}, c_{\text{pb}}, c_{\text{br}}, c_{\text{cf1t}}, c_{\text{cfct}}\}$
\end{itemize}

\subsection{Hendrickson's Implementation}

\begin{lstlisting}
# Carryover affects means:
brmeans[p] <- brmeans[p] + brmeans[p-1] * (1/2)^(tsd/t1half)

# Correlations are FIXED:
correlations[n1,n2] <- modelparam$c.br  # Same regardless of carryover
\end{lstlisting}

The only exception is the biomarker correlation logic (checking if $\mu \neq 0$), which is arguably a modeling artifact rather than a theoretical choice.

\section{Argument 2: Carryover SHOULD Change Correlations}

\subsection{Position}

Carryover creates \textbf{temporal dependency}, which fundamentally alters correlation structure.

\subsection{Reasoning}

\subsubsection{Carryover Creates Mechanistic Correlation}

\begin{itemize}
    \item Without carryover: BR at time $t$ and time $t+1$ are independent given covariates
    \item With carryover: BR at time $t$ \textbf{causally influences} BR at time $t+1$
    \item This should be reflected in the correlation structure
\end{itemize}

\subsubsection{Autocorrelation Should Increase with Carryover}

If effects persist longer, observations are more similar across time:
\begin{itemize}
    \item Strong carryover (long $t_{1/2}$) $\rightarrow$ high temporal correlation
    \item Weak carryover (short $t_{1/2}$) $\rightarrow$ low temporal correlation
\end{itemize}

This is implemented in the alternative approach:
\begin{lstlisting}
carryover_strength <- carryover_halflife / (1 + carryover_halflife)
autocorr_boost <- 0.3 * carryover_strength
adjusted_br <- min(0.95, base_autocorr + autocorr_boost * 1.2)
\end{lstlisting}

\subsubsection{Distinguishing Within-Person vs Between-Person Effects}

\begin{itemize}
    \item Carryover creates \textbf{within-person temporal dependency}
    \item This is distinct from \textbf{between-person trait stability} (random effects)
    \item The correlation structure should capture both
\end{itemize}

\subsubsection{Cross-Time Correlations Matter}

With carryover, TV$_1$, PB$_1$, and BR$_1$ should correlate more strongly with BR$_2$:
\begin{itemize}
    \item Without carryover: weak cross-time correlations
    \item With carryover: stronger cross-time correlations
\end{itemize}

Alternative approach: \texttt{cross\_time\_boost <- 0.4 * carryover\_strength}

\section{What Does the Literature Say?}

\subsection{Time Series / Longitudinal Data Perspective}

Consider an autoregressive model:
\begin{equation}
Y_t = \phi Y_{t-1} + \varepsilon_t
\end{equation}

where:
\begin{itemize}
    \item $\phi$ (autoregression coefficient) $\approx$ carryover effect
    \item Implied correlation: $\text{Cor}(Y_t, Y_{t-k}) = \phi^k$
    \item \textbf{Correlation structure is a CONSEQUENCE of the autoregressive process}
\end{itemize}

This supports the alternative approach: carryover parameter should determine correlation structure.

\subsection{Multivariate Normal Sampling Perspective}

Standard approach:
\begin{equation}
\mathbf{Y} \sim \mathcal{N}(\boldsymbol{\mu}, \boldsymbol{\Sigma})
\end{equation}

where:
\begin{itemize}
    \item $\boldsymbol{\mu}$ captures systematic effects (including carryover)
    \item $\boldsymbol{\Sigma}$ captures residual variability
    \item \textbf{These are independent parameters}
\end{itemize}

This supports Hendrickson's approach: correlations are separate from means.

\section{The Critical Distinction}

\subsection{Hendrickson's Model}

\textbf{Carryover enters ONLY through means:}
\begin{enumerate}
    \item Generate $\mathbf{Y} \sim \mathcal{N}(\boldsymbol{\mu}_{\text{carryover}}, \boldsymbol{\Sigma}_{\text{fixed}})$
    \item Carryover is already ``baked into'' the sampled values
    \item The MVN is a \textbf{single snapshot} -- all time points sampled simultaneously
    \item Temporal dependency is implicit in the mean structure
\end{enumerate}

\textbf{Interpretation:} The covariance matrix represents \textbf{how individuals deviate from their expected trajectory}, not the temporal dependency itself.

\subsection{Alternative Model Enhancement}

\textbf{Carryover enters through BOTH means AND correlations:}
\begin{enumerate}
    \item Generate $\mathbf{Y} \sim \mathcal{N}(\boldsymbol{\mu}_{\text{carryover}}, \boldsymbol{\Sigma}_{\text{carryover}})$
    \item Stronger carryover $\rightarrow$ higher temporal correlations
    \item This makes the \textbf{residual structure} explicitly time-dependent
\end{enumerate}

\textbf{Interpretation:} The covariance matrix represents \textbf{both individual variability AND mechanistic temporal dependency}.

\section{Which is Correct?}

Both approaches are defensible, but they represent \textbf{different modeling philosophies}.

\subsection{Hendrickson's Approach (Means Only) is Correct If:}

You believe carryover is a \textbf{deterministic, population-level phenomenon}:
\begin{itemize}
    \item Everyone's carryover follows $(1/2)^{t/t_{1/2}}$ exactly
    \item Individual variation around this is captured by fixed $c_{\text{br}}$ correlation
    \item The correlation structure represents stable individual traits
\end{itemize}

\textbf{Analogy:} Like a dose-response curve -- everyone follows the same curve, just with different baseline sensitivities.

\subsection{Alternative Approach (Means + Correlations) is Correct If:}

You believe carryover creates \textbf{individual-level temporal dependency}:
\begin{itemize}
    \item People don't all follow $(1/2)^{t/t_{1/2}}$ exactly
    \item There's individual variation in carryover rate
    \item Stronger carryover $\rightarrow$ more ``memory'' in the system $\rightarrow$ higher autocorrelation
\end{itemize}

\textbf{Analogy:} Like a pharmacokinetic model -- elimination rates vary by individual, creating person-specific temporal dependencies.

\section{Assessment}

\subsection{Mathematically/Statistically}

\subsubsection{Hendrickson's Approach is Cleaner:}
\begin{itemize}[label=\textcolor{green}{$\checkmark$}]
    \item Clean separation of mean and covariance
    \item Standard multivariate normal framework
    \item Easier to interpret parameters
    \item Carryover fully captured in mean structure
\end{itemize}

\subsubsection{Alternative Approach is More Mechanistically Accurate:}
\begin{itemize}[label=\textcolor{green}{$\checkmark$}]
    \item Explicitly models temporal dependency induced by carryover
    \item Captures the idea that longer carryover $\rightarrow$ stronger autocorrelation
    \item More aligned with time series thinking
\end{itemize}

However:
\begin{itemize}[label=\textcolor{red}{$\times$}]
    \item \textbf{Potentially double-counting carryover} (once in means, once in correlations)
\end{itemize}

\subsection{The Double-Counting Problem}

\textbf{Critical Issue:} When implementing the alternative approach:

\begin{lstlisting}
# Step 1: Adjust means for carryover
bio_response_means <- apply_carryover_to_component(...)

# Step 2: Adjust correlations for carryover
adjusted_br <- base_autocorr + autocorr_boost * carryover_strength

# Step 3: Sample from MVN
Y ~ MVN(mu_with_carryover, Sigma_with_carryover)
\end{lstlisting}

You might be \textbf{over-representing} carryover effects:
\begin{itemize}
    \item Sampled values inherit temporal dependency from $\boldsymbol{\Sigma}$
    \item Then you ALSO add carryover through $\boldsymbol{\mu}$
    \item Result: Too much persistence?
\end{itemize}

\subsection{Hendrickson's Approach Avoids This}

\begin{lstlisting}
# Step 1: Adjust means for carryover
brmeans[p] <- brmeans[p] + brmeans[p-1] * (1/2)^(t/t_half)

# Step 2: Sample from MVN with FIXED correlations
Y ~ MVN(mu_with_carryover, Sigma_fixed)
\end{lstlisting}

All temporal dependency from carryover is in the means. The correlation captures only \textbf{residual individual-level variation}.

\section{Recommendation}

\subsection{Hendrickson's Approach is Theoretically Cleaner}

\textbf{Why:}

\begin{enumerate}
    \item \textbf{Carryover is already fully specified} in the mean structure through the exponential decay formula
    \item \textbf{Correlation parameters should represent residual variability} after accounting for all systematic effects (treatment, carryover, time trends)
    \item \textbf{Avoids double-counting} temporal dependency
    \item \textbf{Aligns with standard multilevel modeling} where:
    \begin{itemize}
        \item Fixed effects (including carryover) $\rightarrow$ means
        \item Random effects (individual traits) $\rightarrow$ correlations
    \end{itemize}
    \item \textbf{Simpler parameter interpretation} -- $c_{\text{br}}$ means ``individual-level stability'' not ``mechanistic persistence''
\end{enumerate}

\subsection{However, the Alternative Enhancement Could Be Justified If:}

You're trying to model \textbf{individual heterogeneity in carryover rates}:
\begin{itemize}
    \item Some people eliminate drugs faster (short personal $t_{1/2}$)
    \item Some people eliminate slower (long personal $t_{1/2}$)
    \item Population-level carryover\_t1half is the average
    \item Higher correlation represents this individual variation in elimination
\end{itemize}

\textbf{But then you should:}
\begin{enumerate}
    \item Keep the base carryover in means (population average)
    \item Use correlation adjustment to represent \textbf{variance in individual carryover rates}
    \item Be explicit that you're modeling heterogeneous carryover, not just carryover
\end{enumerate}

\section{Conclusion}

\subsection{Answer to the Core Question}

\textbf{Modified means due to carryover should NOT change correlation parameters} in the standard simulation framework because:

\begin{enumerate}
    \item[\textcolor{green}{$\checkmark$}] Carryover is a \textbf{mean effect} (systematic shift in expected value)
    \item[\textcolor{green}{$\checkmark$}] Correlations represent \textbf{residual dependencies} after accounting for means
    \item[\textcolor{green}{$\checkmark$}] Adjusting both creates \textbf{double-counting} of temporal dependency
    \item[\textcolor{green}{$\checkmark$}] Hendrickson's approach (carryover in means only) is the \textbf{standard statistical approach}
\end{enumerate}

\subsection{Implications}

The alternative approach (adjusting both means and correlations) could be valid if reframed as modeling ``individual heterogeneity in carryover rates'' but would need careful theoretical justification to avoid double-counting.

For direct comparison to Hendrickson and standard statistical practice, the recommendation is to \textbf{remove the dynamic correlation adjustment} and keep carryover effects only in the mean structure.

\end{document}
