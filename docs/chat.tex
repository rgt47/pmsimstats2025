% Options for packages loaded elsewhere
\PassOptionsToPackage{unicode}{hyperref}
\PassOptionsToPackage{hyphens}{url}
\documentclass[
]{article}
\usepackage{xcolor}
\usepackage[margin=1in]{geometry}
\usepackage{amsmath,amssymb}
\setcounter{secnumdepth}{5}
\usepackage{iftex}
\ifPDFTeX
  \usepackage[T1]{fontenc}
  \usepackage[utf8]{inputenc}
  \usepackage{textcomp} % provide euro and other symbols
\else % if luatex or xetex
  \usepackage{unicode-math} % this also loads fontspec
  \defaultfontfeatures{Scale=MatchLowercase}
  \defaultfontfeatures[\rmfamily]{Ligatures=TeX,Scale=1}
\fi
\usepackage{lmodern}
\ifPDFTeX\else
  % xetex/luatex font selection
\fi
% Use upquote if available, for straight quotes in verbatim environments
\IfFileExists{upquote.sty}{\usepackage{upquote}}{}
\IfFileExists{microtype.sty}{% use microtype if available
  \usepackage[]{microtype}
  \UseMicrotypeSet[protrusion]{basicmath} % disable protrusion for tt fonts
}{}
\makeatletter
\@ifundefined{KOMAClassName}{% if non-KOMA class
  \IfFileExists{parskip.sty}{%
    \usepackage{parskip}
  }{% else
    \setlength{\parindent}{0pt}
    \setlength{\parskip}{6pt plus 2pt minus 1pt}}
}{% if KOMA class
  \KOMAoptions{parskip=half}}
\makeatother
\usepackage{color}
\usepackage{fancyvrb}
\newcommand{\VerbBar}{|}
\newcommand{\VERB}{\Verb[commandchars=\\\{\}]}
\DefineVerbatimEnvironment{Highlighting}{Verbatim}{commandchars=\\\{\}}
% Add ',fontsize=\small' for more characters per line
\usepackage{framed}
\definecolor{shadecolor}{RGB}{248,248,248}
\newenvironment{Shaded}{\begin{snugshade}}{\end{snugshade}}
\newcommand{\AlertTok}[1]{\textcolor[rgb]{0.94,0.16,0.16}{#1}}
\newcommand{\AnnotationTok}[1]{\textcolor[rgb]{0.56,0.35,0.01}{\textbf{\textit{#1}}}}
\newcommand{\AttributeTok}[1]{\textcolor[rgb]{0.13,0.29,0.53}{#1}}
\newcommand{\BaseNTok}[1]{\textcolor[rgb]{0.00,0.00,0.81}{#1}}
\newcommand{\BuiltInTok}[1]{#1}
\newcommand{\CharTok}[1]{\textcolor[rgb]{0.31,0.60,0.02}{#1}}
\newcommand{\CommentTok}[1]{\textcolor[rgb]{0.56,0.35,0.01}{\textit{#1}}}
\newcommand{\CommentVarTok}[1]{\textcolor[rgb]{0.56,0.35,0.01}{\textbf{\textit{#1}}}}
\newcommand{\ConstantTok}[1]{\textcolor[rgb]{0.56,0.35,0.01}{#1}}
\newcommand{\ControlFlowTok}[1]{\textcolor[rgb]{0.13,0.29,0.53}{\textbf{#1}}}
\newcommand{\DataTypeTok}[1]{\textcolor[rgb]{0.13,0.29,0.53}{#1}}
\newcommand{\DecValTok}[1]{\textcolor[rgb]{0.00,0.00,0.81}{#1}}
\newcommand{\DocumentationTok}[1]{\textcolor[rgb]{0.56,0.35,0.01}{\textbf{\textit{#1}}}}
\newcommand{\ErrorTok}[1]{\textcolor[rgb]{0.64,0.00,0.00}{\textbf{#1}}}
\newcommand{\ExtensionTok}[1]{#1}
\newcommand{\FloatTok}[1]{\textcolor[rgb]{0.00,0.00,0.81}{#1}}
\newcommand{\FunctionTok}[1]{\textcolor[rgb]{0.13,0.29,0.53}{\textbf{#1}}}
\newcommand{\ImportTok}[1]{#1}
\newcommand{\InformationTok}[1]{\textcolor[rgb]{0.56,0.35,0.01}{\textbf{\textit{#1}}}}
\newcommand{\KeywordTok}[1]{\textcolor[rgb]{0.13,0.29,0.53}{\textbf{#1}}}
\newcommand{\NormalTok}[1]{#1}
\newcommand{\OperatorTok}[1]{\textcolor[rgb]{0.81,0.36,0.00}{\textbf{#1}}}
\newcommand{\OtherTok}[1]{\textcolor[rgb]{0.56,0.35,0.01}{#1}}
\newcommand{\PreprocessorTok}[1]{\textcolor[rgb]{0.56,0.35,0.01}{\textit{#1}}}
\newcommand{\RegionMarkerTok}[1]{#1}
\newcommand{\SpecialCharTok}[1]{\textcolor[rgb]{0.81,0.36,0.00}{\textbf{#1}}}
\newcommand{\SpecialStringTok}[1]{\textcolor[rgb]{0.31,0.60,0.02}{#1}}
\newcommand{\StringTok}[1]{\textcolor[rgb]{0.31,0.60,0.02}{#1}}
\newcommand{\VariableTok}[1]{\textcolor[rgb]{0.00,0.00,0.00}{#1}}
\newcommand{\VerbatimStringTok}[1]{\textcolor[rgb]{0.31,0.60,0.02}{#1}}
\newcommand{\WarningTok}[1]{\textcolor[rgb]{0.56,0.35,0.01}{\textbf{\textit{#1}}}}
\usepackage{longtable,booktabs,array}
\newcounter{none} % for unnumbered tables
\usepackage{calc} % for calculating minipage widths
% Correct order of tables after \paragraph or \subparagraph
\usepackage{etoolbox}
\makeatletter
\patchcmd\longtable{\par}{\if@noskipsec\mbox{}\fi\par}{}{}
\makeatother
% Allow footnotes in longtable head/foot
\IfFileExists{footnotehyper.sty}{\usepackage{footnotehyper}}{\usepackage{footnote}}
\makesavenoteenv{longtable}
\usepackage{graphicx}
\makeatletter
\newsavebox\pandoc@box
\newcommand*\pandocbounded[1]{% scales image to fit in text height/width
  \sbox\pandoc@box{#1}%
  \Gscale@div\@tempa{\textheight}{\dimexpr\ht\pandoc@box+\dp\pandoc@box\relax}%
  \Gscale@div\@tempb{\linewidth}{\wd\pandoc@box}%
  \ifdim\@tempb\p@<\@tempa\p@\let\@tempa\@tempb\fi% select the smaller of both
  \ifdim\@tempa\p@<\p@\scalebox{\@tempa}{\usebox\pandoc@box}%
  \else\usebox{\pandoc@box}%
  \fi%
}
% Set default figure placement to htbp
\def\fps@figure{htbp}
\makeatother
\setlength{\emergencystretch}{3em} % prevent overfull lines
\providecommand{\tightlist}{%
  \setlength{\itemsep}{0pt}\setlength{\parskip}{0pt}}
\usepackage{bookmark}
\IfFileExists{xurl.sty}{\usepackage{xurl}}{} % add URL line breaks if available
\urlstyle{same}
\hypersetup{
  pdftitle={Comparing Classic Crossover Trials and Aggregated N-of-1 Trials},
  hidelinks,
  pdfcreator={LaTeX via pandoc}}

\title{Comparing Classic Crossover Trials and Aggregated N-of-1 Trials}
\usepackage{etoolbox}
\makeatletter
\providecommand{\subtitle}[1]{% add subtitle to \maketitle
  \apptocmd{\@title}{\par {\large #1 \par}}{}{}
}
\makeatother
\subtitle{A Methodological White Paper}
\author{}
\date{\vspace{-2.5em}2025-11-21}

\begin{document}
\maketitle

{
\setcounter{tocdepth}{3}
\tableofcontents
}
\section{Abstract}\label{abstract}

Classic crossover trials and aggregated N-of-1 trials share surface
similarities, including repeated administration of treatments within
individuals and frequent use of mixed-effects models. However, despite
this overlap, the designs differ fundamentally in structure, inferential
goals, estimands, replication, and the appropriate specification of
linear mixed-effects models. This white paper expands on the conceptual
and statistical distinctions between classic crossover designs and
aggregated N-of-1 designs, includes narrative citations throughout, and
provides a worked example involving a hypothetical PTSD trial using the
Clinician-Administered PTSD Scale (CAPS) to illustrate how either design
might be implemented.

\section{Introduction}\label{introduction}

Repeated-treatment clinical trial designs reduce between-person
variability by allowing each participant to serve as their own control.
The best-known version of this approach is the classic crossover trial,
widely used in chronic and stable conditions (Jones \& Kenward, 2014).
In contrast, N-of-1 trials, especially aggregated N-of-1 trials, are
designed to evaluate individual-level responses and heterogeneity
(Lillie et al., 2011; Guyatt et al., 2000). Despite frequent use of
mixed-effects models in both designs, the correct statistical model
depends on the inferential target and replication structure (Senn,
2002).

This white paper expands on how these two designs differ and why
mixed-model analyses cannot be identical, even though they share an
analytical framework. It concludes with an applied example involving a
PTSD treatment trial using CAPS scores.

\section{Background}\label{background}

\subsection{Classic Crossover Trials}\label{classic-crossover-trials}

Classic crossover trials are structured in a small number of periods
(typically 2--4), with treatment sequences such as AB/BA or ABBA (Araujo
\& Julious, 2014). Their primary purpose is to estimate a population
average treatment effect while controlling for period and sequence
effects (Jones \& Kenward, 2014). These designs assume relative
stability over time, limited carryover, and generally homogeneous
treatment effects across participants (Senn, 2002).

\subsection{N-of-1 Trials}\label{n-of-1-trials}

N-of-1 trials constitute repeated, randomized, within-person experiments
using alternating treatment periods (Guyatt et al., 2000). They provide
dense, within-person data suitable for estimating individual treatment
effects, often with multiple AB pairs (e.g., ABABAB).

\subsection{Aggregated N-of-1 Trials}\label{aggregated-n-of-1-trials}

Aggregated N-of-1 studies combine individual trials through hierarchical
modeling to estimate both person-specific and population-level effects
(Zucker et al., 2010; Punja et al., 2016). They explicitly target
treatment-effect heterogeneity and leverage partial pooling (Deaton \&
Cartwright, 2018).

\subsection{Hybrid Designs}\label{hybrid-designs}

Between classic crossover and full N-of-1 designs lies a spectrum of
hybrid approaches. These designs extend crossover trials to 3--4
periods, allowing some within-person replication while maintaining
feasibility. For example, a 4-path randomization (ABAB, ABBA, BAAB,
BABA) provides two treatment contrasts per person---insufficient for
robust individual effect estimation but enough to partially estimate
heterogeneity (Senn, 2002). Hybrid designs represent a practical
compromise when full N-of-1 density is infeasible but pure crossover
replication is inadequate.

\section{Core Differences Between
Designs}\label{core-differences-between-designs}

\subsection{Purpose and Estimands}\label{purpose-and-estimands}

The essential difference lies in the estimands, not the analytic
framework.

\begin{itemize}
\tightlist
\item
  \textbf{Classic crossover}: population mean treatment effect (Jones \&
  Kenward, 2014).
\item
  \textbf{Aggregated N-of-1}: individual treatment effects plus
  heterogeneity (Lillie et al., 2011).
\end{itemize}

\subsection{Replication Structure}\label{replication-structure}

Replication determines what can be estimated.

\begin{itemize}
\tightlist
\item
  Crossover trials provide limited within-person replication---often
  only one treatment contrast---making random slopes for treatment
  unidentifiable (Brown, 1980).
\item
  Aggregated N-of-1 trials provide substantial within-person
  replication, enabling estimation of random slopes and heterogeneity.
\end{itemize}

\subsection{Homogeneity vs
Heterogeneity}\label{homogeneity-vs-heterogeneity}

Crossover designs traditionally assume minimal heterogeneity (Senn,
2002). Aggregated N-of-1 designs assume and model heterogeneity directly
(Punja et al., 2016).

\subsection{Carryover Effects}\label{carryover-effects}

Carryover---the persistence of treatment effects into subsequent
periods---is handled fundamentally differently across designs:

\begin{itemize}
\item
  \textbf{Classic crossover}: Carryover is typically tested and assumed
  negligible. If present, it confounds period and treatment effects,
  potentially invalidating the design. Washout periods are employed to
  eliminate carryover, and statistical tests for carryover are often
  underpowered (Jones \& Kenward, 2014; Senn, 2002).
\item
  \textbf{Aggregated N-of-1}: Multiple treatment cycles enable explicit
  modeling of carryover as a parameter. Carryover can be incorporated as
  a fixed effect based on prior-period treatment, with magnitude
  determined by pharmacokinetic half-life. This transforms carryover
  from a nuisance to an estimable quantity.
\item
  \textbf{Hybrid designs}: With 3--4 periods, some carryover estimation
  becomes possible, though less robust than in full N-of-1 designs.
\end{itemize}

The key distinction: crossover designs require carryover to be absent,
while aggregated N-of-1 designs can accommodate and estimate it.

\subsection{Biomarker-Treatment
Interactions}\label{biomarker-treatment-interactions}

A critical extension beyond average treatment effects is the estimation
of treatment effect modification by baseline or time-varying biomarkers.
This addresses the personalized medicine question: ``For whom does this
treatment work best?''

\begin{itemize}
\item
  \textbf{Classic crossover}: Can include baseline biomarker × treatment
  interactions as fixed effects. However, with limited within-person
  replication, only population-level effect modification is estimable.
  The model answers: ``Do patients with high biomarker values respond
  differently on average?''
\item
  \textbf{Aggregated N-of-1}: Dense within-person data enables
  estimation of both:

  \begin{enumerate}
  \def\labelenumi{\arabic{enumi}.}
  \tightlist
  \item
    Population-level biomarker × treatment interactions
  \item
    Individual-specific biomarker × treatment slopes (random slopes)
  \end{enumerate}

  This allows the model to answer: ``How does this specific patient's
  biomarker value predict their treatment response?'' Time-varying
  biomarkers can be incorporated, linking within-person biomarker
  fluctuations to within-person treatment response variation.
\item
  \textbf{Hybrid designs}: Can estimate population-level biomarker ×
  treatment interactions with some precision, but individual-specific
  interaction terms remain difficult to identify.
\end{itemize}

The inclusion of biomarker interactions shifts the estimand from ``Does
treatment work?'' to ``For whom and under what conditions does treatment
work?''---a fundamentally different scientific question that aggregated
N-of-1 designs are uniquely positioned to answer.

\section{Mixed-Model Specification}\label{mixed-model-specification}

\subsection{Classic Crossover Mixed
Model}\label{classic-crossover-mixed-model}

A typical model (where \(i\) indexes participants, \(t\) indexes
periods):

\[Y_{it} = \beta_0 + \beta_{treat} \times Treat_{it} + \beta_{period} \times Period_t + \beta_{bm} \times BM_i + \beta_{int} \times (Treat_{it} \times BM_i) + u_i + \varepsilon_{it}\]

\begin{itemize}
\tightlist
\item
  Random intercept only (\(u_i\))
\item
  Treatment effect treated as fixed (Araujo \& Julious, 2014)
\item
  Biomarker × treatment interaction estimable at population level
\item
  Random slopes for treatment generally not identifiable due to
  insufficient replication (Brown, 1980)
\end{itemize}

\subsection{Aggregated N-of-1 Hierarchical Mixed
Model}\label{aggregated-n-of-1-hierarchical-mixed-model}

\[Y_{it} = \beta_0 + (\beta_{treat} + u_{i,treat}) \times Treat_{it} + \beta_{period} \times Period_t + \beta_{bm} \times BM_{it} + (\beta_{int} + u_{i,int}) \times (Treat_{it} \times BM_{it}) + u_i + \varepsilon_{it}\]

\begin{itemize}
\tightlist
\item
  Random intercept (\(u_i\)), random slope for treatment
  (\(u_{i,treat}\)), and optionally random slope for biomarker
  interaction (\(u_{i,int}\))
\item
  Individual treatment effect = \(\beta_{treat} + u_{i,treat}\)
\item
  Individual biomarker interaction = \(\beta_{int} + u_{i,int}\)
\item
  Heterogeneity quantified via \(Var(u_{i,treat})\) and
  \(Var(u_{i,int})\)
\item
  Time-varying biomarkers (\(BM_{it}\)) can link within-person biomarker
  changes to within-person treatment response
\end{itemize}

\subsection{Hybrid Design Mixed Model}\label{hybrid-design-mixed-model}

\[Y_{it} = \beta_0 + \beta_{treat} \times Treat_{it} + \beta_{period} \times Period_t + \beta_{bm} \times BM_i + \beta_{int} \times (Treat_{it} \times BM_i) + \beta_{carry} \times Carry_{it} + u_i + \varepsilon_{it}\]

\begin{itemize}
\tightlist
\item
  Random intercept with fixed effects for treatment, biomarker
  interaction, and carryover
\item
  With 3--4 periods, limited random slope estimation may be attempted
  but often yields singular fits
\item
  Carryover explicitly modeled as fixed effect based on prior-period
  treatment
\end{itemize}

\subsection{Why the Models Cannot Be the
Same}\label{why-the-models-cannot-be-the-same}

Even though both use LMMs, the models differ because of:

\begin{itemize}
\tightlist
\item
  Different estimands
\item
  Different random-effect structures
\item
  Different identifiability conditions
\item
  Different inferential goals (Senn, 2002; Zucker et al., 2010)
\end{itemize}

\section{Example: A PTSD (CAPS) Trial Designed as Either Crossover or
Aggregated
N-of-1}\label{example-a-ptsd-caps-trial-designed-as-either-crossover-or-aggregated-n-of-1}

To illustrate the design differences, consider a hypothetical new
fast-acting pharmacologic agent for reducing PTSD symptoms, measured
using CAPS-5.

\subsection{Clinical Assumptions}\label{clinical-assumptions}

\begin{itemize}
\tightlist
\item
  Onset: 24--48 hours
\item
  Washout: 48--72 hours
\item
  Outcome measurement: daily CAPS or EMA-based PTSD symptom indices
\item
  Condition: chronic, variable symptoms, suitable for repeated
  within-person comparisons
\end{itemize}

\subsection{Option 1: Classic 2-Period Crossover
Design}\label{option-1-classic-2-period-crossover-design}

\textbf{Design:}

\begin{itemize}
\tightlist
\item
  Sequence AB or BA
\item
  Treatment periods: 4 weeks each
\item
  Washout: 1 week
\item
  N ≈ 40 participants
\end{itemize}

\textbf{Strengths:}

\begin{itemize}
\tightlist
\item
  Well-aligned with regulatory expectations
\item
  Simple interpretation of average treatment effect
\end{itemize}

\textbf{Limitations:}

\begin{itemize}
\tightlist
\item
  Only one treatment contrast per person
\item
  Cannot estimate individual treatment effects
\item
  Less suitable if heterogeneity is clinically important (Jones \&
  Kenward, 2014)
\end{itemize}

\textbf{Mixed Model:}

\begin{itemize}
\tightlist
\item
  Random intercept only
\item
  Fixed treatment effect
\item
  Period and sequence as fixed effects
\end{itemize}

\subsection{Option 2: Aggregated N-of-1
Trial}\label{option-2-aggregated-n-of-1-trial}

\textbf{Design:}

\begin{itemize}
\tightlist
\item
  Each participant undergoes 6 cycles of active vs placebo: ABABAB (or
  BAABAB, individualized randomization)
\item
  Treatment periods: 1 week active, 1 week placebo
\item
  Washout: same-week washout built in due to rapid clearance
\item
  CAPS measured daily using EMA or twice weekly in clinic
\item
  N = 20 participants (fewer needed due to dense data)
\end{itemize}

\subsubsection{Why Fewer Participants
Suffice}\label{why-fewer-participants-suffice}

The N-of-1 design achieves adequate power with fewer participants
because:

\begin{enumerate}
\def\labelenumi{\arabic{enumi}.}
\item
  \textbf{Increased effective sample size}: Each participant contributes
  multiple treatment contrasts (6 cycles = 6 contrasts vs 1 contrast in
  crossover). With 20 participants × 6 cycles, the effective number of
  treatment comparisons approaches 120.
\item
  \textbf{Reduced within-person variance}: Dense repeated measurements
  within periods average out measurement error and day-to-day
  fluctuations.
\item
  \textbf{Partial pooling}: Hierarchical models borrow strength across
  participants. Participants with fewer observations or more noise are
  shrunk toward the population mean, improving precision for both
  individual and population estimates (Zucker et al., 2010).
\item
  \textbf{Direct estimation target}: If the goal is individual treatment
  effects (not just population average), N-of-1 provides the data
  structure required---crossover designs cannot achieve this regardless
  of sample size.
\end{enumerate}

\textbf{Strengths:}

\begin{itemize}
\tightlist
\item
  Estimates individual treatment effects
\item
  Captures treatment-effect heterogeneity
\item
  Estimates biomarker × treatment interactions at individual level
\item
  Can explicitly model carryover effects
\item
  Partially pooled model improves precision (Lillie et al., 2011; Zucker
  et al., 2010)
\end{itemize}

\textbf{Limitations:}

\begin{itemize}
\tightlist
\item
  Higher participant burden
\item
  Requires stable condition and rapid washout
\item
  More complex data management
\end{itemize}

\textbf{Mixed Model:}

\begin{itemize}
\tightlist
\item
  Random intercept and random slope for treatment
\item
  Biomarker × treatment interaction (fixed or random)
\item
  Carryover effect (fixed)
\item
  Individual effects reported alongside pooled effect
\item
  Heterogeneity explicitly estimated
\end{itemize}

\subsection{Interpretation
Differences}\label{interpretation-differences}

{\def\LTcaptype{none} % do not increment counter
\begin{longtable}[]{@{}
  >{\raggedright\arraybackslash}p{(\linewidth - 6\tabcolsep) * \real{0.2083}}
  >{\raggedright\arraybackslash}p{(\linewidth - 6\tabcolsep) * \real{0.2292}}
  >{\raggedright\arraybackslash}p{(\linewidth - 6\tabcolsep) * \real{0.1667}}
  >{\raggedright\arraybackslash}p{(\linewidth - 6\tabcolsep) * \real{0.3958}}@{}}
\toprule\noalign{}
\begin{minipage}[b]{\linewidth}\raggedright
Question
\end{minipage} & \begin{minipage}[b]{\linewidth}\raggedright
Crossover
\end{minipage} & \begin{minipage}[b]{\linewidth}\raggedright
Hybrid
\end{minipage} & \begin{minipage}[b]{\linewidth}\raggedright
Aggregated N-of-1
\end{minipage} \\
\midrule\noalign{}
\endhead
\bottomrule\noalign{}
\endlastfoot
Does the drug work on average? & Yes & Yes & Yes \\
Does the drug work for this individual? & No & Limited & Yes \\
Is treatment effect heterogeneous? & Rarely estimable & Partially &
Yes \\
Does biomarker predict response (population)? & Yes & Yes & Yes \\
Does biomarker predict response (individual)? & No & No & Yes \\
Can carryover be estimated? & No (assumed absent) & Partially & Yes \\
Does each person have replicated comparisons? & No & Limited (2-3) & Yes
(6+) \\
\end{longtable}
}

\subsection{Choosing Between the Designs for a PTSD
Trial}\label{choosing-between-the-designs-for-a-ptsd-trial}

\begin{itemize}
\tightlist
\item
  If the goal is regulatory approval or demonstration of a
  population-average effect → Crossover (Jones \& Kenward, 2014).
\item
  If the goal is to understand who responds, optimize personalized
  treatment, or examine heterogeneity → Aggregated N-of-1 (Lillie et
  al., 2011; Punja et al., 2016).
\item
  If measurement is frequent and the drug acts quickly → N-of-1 is
  advantageous.
\item
  If measurement is sporadic or drug has slow dynamics → Crossover
  preferred.
\end{itemize}

\section{Conclusion}\label{conclusion}

Although classic crossover and aggregated N-of-1 designs can both be
analyzed using mixed-effects models, the correct models differ
fundamentally. Crossover trials typically feature random-intercept
models targeting a population-level treatment effect with minimal
within-person replication. Aggregated N-of-1 designs feature
hierarchical mixed models with random slopes for treatment, enabling
estimation of individual effects and treatment heterogeneity. Hybrid
designs occupy a middle ground, extending crossover to 3--4 periods for
some heterogeneity estimation.

A critical extension is biomarker × treatment interactions. All three
designs can estimate population-level effect modification (does
biomarker predict response on average?), but only aggregated N-of-1
designs provide sufficient within-person replication to estimate
individual-specific biomarker interactions. This shifts the estimand
from ``Does treatment work?'' to ``For whom and under what conditions
does treatment work?''---the fundamental question of personalized
medicine.

Similarly, carryover effects---typically a nuisance to be eliminated in
crossover designs---can be explicitly modeled and estimated in
aggregated N-of-1 designs due to multiple treatment cycles.

The choice of design should be guided by:

\begin{itemize}
\tightlist
\item
  \textbf{Scientific aims}: Population average vs individual-level
  inference
\item
  \textbf{Estimand}: Average treatment effect vs biomarker-stratified
  effects vs individual effects
\item
  \textbf{Pharmacologic properties}: Onset, washout, and carryover
  characteristics
\item
  \textbf{Feasibility}: Participant burden, measurement frequency, study
  duration
\end{itemize}

When the goal is regulatory approval of a population-average effect,
crossover designs remain appropriate. When the goal is precision
medicine---identifying who responds and why---aggregated N-of-1 designs
are uniquely positioned to answer these questions.

\section{References}\label{references}

\begin{itemize}
\tightlist
\item
  Araujo, A., \& Julious, S. A. (2014). Understanding the assumptions of
  crossover trials. \emph{Pharmaceutical Statistics}, 13(6), 341--350.
\item
  Brown, H. (1980). The analysis of variance and covariance in crossover
  trials. \emph{Biometrics}, 36(1), 69--79.
\item
  Deaton, A., \& Cartwright, N. (2018). Understanding and
  misunderstanding randomized controlled trials. \emph{Social Science \&
  Medicine}, 210, 2--21.
\item
  Guyatt, G. H., et al.~(2000). The N-of-1 randomized controlled trial:
  Clinical usefulness. \emph{Annals of Internal Medicine}, 112(4),
  293--299.
\item
  Jones, B., \& Kenward, M. G. (2014). \emph{Design and Analysis of
  Cross-Over Trials} (3rd ed.). Chapman \& Hall/CRC.
\item
  Lillie, E. O., et al.~(2011). The n-of-1 clinical trial: The ultimate
  strategy for individualizing medicine? \emph{Personalized Medicine},
  8(2), 161--173.
\item
  Punja, S., et al.~(2016). N-of-1 trials for precision medicine: A
  systematic review. \emph{Journal of Clinical Epidemiology}, 76, 1--13.
\item
  Schmid, C. H., et al.~(2013). Effect of statin therapy on muscle
  symptoms: An individual patient data meta-analysis. \emph{JAMA
  Internal Medicine}, 173(16), 1--9.
\item
  Senn, S. (2002). \emph{Cross-over Trials in Clinical Research} (2nd
  ed.). Wiley.
\item
  Zucker, D. R., Ruthazer, R., \& Schmid, C. H. (2010). Individual
  (N-of-1) trials can be combined to give population comparative
  treatment effect estimates: Methodologic considerations. \emph{Journal
  of Clinical Epidemiology}, 63(12), 1312--1323.
\end{itemize}

\newpage

\section{Appendix: R Code for Simulation and
Analysis}\label{appendix-r-code-for-simulation-and-analysis}

Below are example R code chunks to simulate datasets for a classic
crossover and an aggregated N-of-1 design, and to fit appropriate
mixed-effects models. The code uses \texttt{lme4} for frequentist mixed
models, \texttt{nlme} for correlation structures if desired, and
\texttt{brms} for a Bayesian hierarchical alternative.

\begin{Shaded}
\begin{Highlighting}[]
\FunctionTok{library}\NormalTok{(tidyverse)}
\FunctionTok{library}\NormalTok{(lme4)}
\FunctionTok{library}\NormalTok{(nlme)}
\FunctionTok{library}\NormalTok{(emmeans)}
\CommentTok{\# library(brms)  \# Uncomment if using Bayesian models}
\end{Highlighting}
\end{Shaded}

\subsection{Simulate a Classic 2-Period Crossover Dataset with
Biomarker}\label{simulate-a-classic-2-period-crossover-dataset-with-biomarker}

\begin{Shaded}
\begin{Highlighting}[]
\FunctionTok{set.seed}\NormalTok{(}\DecValTok{2025}\NormalTok{)}

\CommentTok{\# Parameters}
\NormalTok{n\_subj }\OtherTok{\textless{}{-}} \DecValTok{80}
\NormalTok{periods }\OtherTok{\textless{}{-}} \DecValTok{2}

\CommentTok{\# True effects}
\NormalTok{mu }\OtherTok{\textless{}{-}} \DecValTok{20}                    \CommentTok{\# baseline mean CAPS score}
\NormalTok{beta\_treat }\OtherTok{\textless{}{-}} \SpecialCharTok{{-}}\DecValTok{4}            \CommentTok{\# average treatment effect (reduction)}
\NormalTok{beta\_bm }\OtherTok{\textless{}{-}} \DecValTok{2}                \CommentTok{\# biomarker main effect}
\NormalTok{beta\_int }\OtherTok{\textless{}{-}} \SpecialCharTok{{-}}\FloatTok{1.5}            \CommentTok{\# biomarker x treatment interaction}
\NormalTok{beta\_period }\OtherTok{\textless{}{-}} \FloatTok{0.5}          \CommentTok{\# period effect}
\NormalTok{sigma\_subj }\OtherTok{\textless{}{-}} \DecValTok{6}             \CommentTok{\# between{-}subject SD}
\NormalTok{sigma\_resid }\OtherTok{\textless{}{-}} \DecValTok{8}            \CommentTok{\# residual SD}

\CommentTok{\# Create subject{-}level intercepts and baseline biomarker}
\NormalTok{subj\_df }\OtherTok{\textless{}{-}} \FunctionTok{tibble}\NormalTok{(}
  \AttributeTok{subject =} \DecValTok{1}\SpecialCharTok{:}\NormalTok{n\_subj,}
  \AttributeTok{u =} \FunctionTok{rnorm}\NormalTok{(n\_subj, }\DecValTok{0}\NormalTok{, sigma\_subj),}
  \AttributeTok{bm =} \FunctionTok{rnorm}\NormalTok{(n\_subj, }\DecValTok{0}\NormalTok{, }\DecValTok{1}\NormalTok{)}
\NormalTok{)}

\CommentTok{\# Assign half to sequence AB and half to BA}
\NormalTok{subj\_df }\OtherTok{\textless{}{-}}\NormalTok{ subj\_df }\SpecialCharTok{\%\textgreater{}\%}
  \FunctionTok{mutate}\NormalTok{(}\AttributeTok{sequence =} \FunctionTok{rep}\NormalTok{(}\FunctionTok{c}\NormalTok{(}\StringTok{"AB"}\NormalTok{, }\StringTok{"BA"}\NormalTok{), }\AttributeTok{length.out =}\NormalTok{ n\_subj))}

\CommentTok{\# Expand to periods}
\NormalTok{crossover }\OtherTok{\textless{}{-}}\NormalTok{ subj\_df }\SpecialCharTok{\%\textgreater{}\%}
  \FunctionTok{group\_by}\NormalTok{(subject) }\SpecialCharTok{\%\textgreater{}\%}
  \FunctionTok{reframe}\NormalTok{(}\AttributeTok{period =} \DecValTok{1}\SpecialCharTok{:}\NormalTok{periods, }\AttributeTok{u =}\NormalTok{ u, }\AttributeTok{bm =}\NormalTok{ bm, }\AttributeTok{sequence =}\NormalTok{ sequence) }\SpecialCharTok{\%\textgreater{}\%}
  \FunctionTok{mutate}\NormalTok{(}
    \AttributeTok{treatment =} \FunctionTok{case\_when}\NormalTok{(}
\NormalTok{      sequence }\SpecialCharTok{==} \StringTok{"AB"} \SpecialCharTok{\&}\NormalTok{ period }\SpecialCharTok{==} \DecValTok{1} \SpecialCharTok{\textasciitilde{}} \StringTok{"A"}\NormalTok{,}
\NormalTok{      sequence }\SpecialCharTok{==} \StringTok{"AB"} \SpecialCharTok{\&}\NormalTok{ period }\SpecialCharTok{==} \DecValTok{2} \SpecialCharTok{\textasciitilde{}} \StringTok{"B"}\NormalTok{,}
\NormalTok{      sequence }\SpecialCharTok{==} \StringTok{"BA"} \SpecialCharTok{\&}\NormalTok{ period }\SpecialCharTok{==} \DecValTok{1} \SpecialCharTok{\textasciitilde{}} \StringTok{"B"}\NormalTok{,}
\NormalTok{      sequence }\SpecialCharTok{==} \StringTok{"BA"} \SpecialCharTok{\&}\NormalTok{ period }\SpecialCharTok{==} \DecValTok{2} \SpecialCharTok{\textasciitilde{}} \StringTok{"A"}
\NormalTok{    ),}
    \AttributeTok{trt\_indicator =} \FunctionTok{if\_else}\NormalTok{(treatment }\SpecialCharTok{==} \StringTok{"B"}\NormalTok{, }\DecValTok{1}\NormalTok{, }\DecValTok{0}\NormalTok{)}
\NormalTok{  )}

\CommentTok{\# Simulate outcome with biomarker interaction}
\NormalTok{crossover }\OtherTok{\textless{}{-}}\NormalTok{ crossover }\SpecialCharTok{\%\textgreater{}\%}
  \FunctionTok{mutate}\NormalTok{(}
    \AttributeTok{Y =}\NormalTok{ mu }\SpecialCharTok{+}\NormalTok{ u }\SpecialCharTok{+}
\NormalTok{      beta\_period }\SpecialCharTok{*}\NormalTok{ period }\SpecialCharTok{+}
\NormalTok{      beta\_treat }\SpecialCharTok{*}\NormalTok{ trt\_indicator }\SpecialCharTok{+}
\NormalTok{      beta\_bm }\SpecialCharTok{*}\NormalTok{ bm }\SpecialCharTok{+}
\NormalTok{      beta\_int }\SpecialCharTok{*}\NormalTok{ trt\_indicator }\SpecialCharTok{*}\NormalTok{ bm }\SpecialCharTok{+}
      \FunctionTok{rnorm}\NormalTok{(}\FunctionTok{n}\NormalTok{(), }\DecValTok{0}\NormalTok{, sigma\_resid)}
\NormalTok{  )}

\CommentTok{\# Quick glance}
\FunctionTok{head}\NormalTok{(crossover)}
\CommentTok{\#\textgreater{} \# A tibble: 6 x 8}
\CommentTok{\#\textgreater{}   subject period     u     bm sequence treatment trt\_indicator     Y}
\CommentTok{\#\textgreater{}     \textless{}int\textgreater{}  \textless{}int\textgreater{} \textless{}dbl\textgreater{}  \textless{}dbl\textgreater{} \textless{}chr\textgreater{}    \textless{}chr\textgreater{}             \textless{}dbl\textgreater{} \textless{}dbl\textgreater{}}
\CommentTok{\#\textgreater{} 1       1      1 3.72  {-}1.12  AB       A                     0 24.2 }
\CommentTok{\#\textgreater{} 2       1      2 3.72  {-}1.12  AB       B                     1 35.4 }
\CommentTok{\#\textgreater{} 3       2      1 0.214  1.47  BA       B                     1 16.7 }
\CommentTok{\#\textgreater{} 4       2      2 0.214  1.47  BA       A                     0 23.4 }
\CommentTok{\#\textgreater{} 5       3      1 4.64   0.205 AB       A                     0 18.6 }
\CommentTok{\#\textgreater{} 6       3      2 4.64   0.205 AB       B                     1  8.08}
\end{Highlighting}
\end{Shaded}

\subsection{Fit the Classic Crossover Mixed Model with Biomarker
Interaction}\label{fit-the-classic-crossover-mixed-model-with-biomarker-interaction}

\begin{Shaded}
\begin{Highlighting}[]
\CommentTok{\# lmer with random intercept and biomarker x treatment interaction}
\NormalTok{m\_crossover }\OtherTok{\textless{}{-}} \FunctionTok{lmer}\NormalTok{(}
\NormalTok{  Y }\SpecialCharTok{\textasciitilde{}}\NormalTok{ trt\_indicator }\SpecialCharTok{*}\NormalTok{ bm }\SpecialCharTok{+} \FunctionTok{factor}\NormalTok{(period) }\SpecialCharTok{+}\NormalTok{ (}\DecValTok{1} \SpecialCharTok{|}\NormalTok{ subject),}
  \AttributeTok{data =}\NormalTok{ crossover}
\NormalTok{)}
\FunctionTok{summary}\NormalTok{(m\_crossover)}
\CommentTok{\#\textgreater{} Linear mixed model fit by REML [\textquotesingle{}lmerMod\textquotesingle{}]}
\CommentTok{\#\textgreater{} Formula: Y \textasciitilde{} trt\_indicator * bm + factor(period) + (1 | subject)}
\CommentTok{\#\textgreater{}    Data: crossover}
\CommentTok{\#\textgreater{} }
\CommentTok{\#\textgreater{} REML criterion at convergence: 1174.6}
\CommentTok{\#\textgreater{} }
\CommentTok{\#\textgreater{} Scaled residuals: }
\CommentTok{\#\textgreater{}      Min       1Q   Median       3Q      Max }
\CommentTok{\#\textgreater{} {-}2.87787 {-}0.58310  0.08426  0.63764  1.77324 }
\CommentTok{\#\textgreater{} }
\CommentTok{\#\textgreater{} Random effects:}
\CommentTok{\#\textgreater{}  Groups   Name        Variance Std.Dev.}
\CommentTok{\#\textgreater{}  subject  (Intercept) 34.53    5.876   }
\CommentTok{\#\textgreater{}  Residual             71.07    8.431   }
\CommentTok{\#\textgreater{} Number of obs: 160, groups:  subject, 80}
\CommentTok{\#\textgreater{} }
\CommentTok{\#\textgreater{} Fixed effects:}
\CommentTok{\#\textgreater{}                  Estimate Std. Error t value}
\CommentTok{\#\textgreater{} (Intercept)        20.321      1.360  14.940}
\CommentTok{\#\textgreater{} trt\_indicator      {-}3.328      1.358  {-}2.451}
\CommentTok{\#\textgreater{} bm                  2.493      1.229   2.028}
\CommentTok{\#\textgreater{} factor(period)2     1.378      1.348   1.022}
\CommentTok{\#\textgreater{} trt\_indicator:bm   {-}3.757      1.437  {-}2.615}
\CommentTok{\#\textgreater{} }
\CommentTok{\#\textgreater{} Correlation of Fixed Effects:}
\CommentTok{\#\textgreater{}             (Intr) trt\_nd bm     fct()2}
\CommentTok{\#\textgreater{} trt\_indictr {-}0.513                     }
\CommentTok{\#\textgreater{} bm           0.207 {-}0.112              }
\CommentTok{\#\textgreater{} factr(prd)2 {-}0.510  0.029 {-}0.088       }
\CommentTok{\#\textgreater{} trt\_ndctr:b {-}0.170  0.191 {-}0.585  0.151}

\CommentTok{\# Estimated average treatment effect (at mean biomarker = 0)}
\FunctionTok{fixef}\NormalTok{(m\_crossover)[}\StringTok{"trt\_indicator"}\NormalTok{]}
\CommentTok{\#\textgreater{} trt\_indicator }
\CommentTok{\#\textgreater{}     {-}3.328289}

\CommentTok{\# Estimated biomarker x treatment interaction (population{-}level)}
\FunctionTok{fixef}\NormalTok{(m\_crossover)[}\StringTok{"trt\_indicator:bm"}\NormalTok{]}
\CommentTok{\#\textgreater{} trt\_indicator:bm }
\CommentTok{\#\textgreater{}         {-}3.75727}

\CommentTok{\# Treatment effect at different biomarker levels}
\FunctionTok{emmeans}\NormalTok{(m\_crossover, }\SpecialCharTok{\textasciitilde{}}\NormalTok{ trt\_indicator }\SpecialCharTok{|}\NormalTok{ bm, }\AttributeTok{at =} \FunctionTok{list}\NormalTok{(}\AttributeTok{bm =} \FunctionTok{c}\NormalTok{(}\SpecialCharTok{{-}}\DecValTok{1}\NormalTok{, }\DecValTok{0}\NormalTok{, }\DecValTok{1}\NormalTok{)))}
\CommentTok{\#\textgreater{} bm = {-}1:}
\CommentTok{\#\textgreater{}  trt\_indicator emmean   SE  df lower.CL upper.CL}
\CommentTok{\#\textgreater{}              0   18.5 1.53 141     15.5     21.5}
\CommentTok{\#\textgreater{}              1   18.9 1.53 141     15.9     22.0}
\CommentTok{\#\textgreater{} }
\CommentTok{\#\textgreater{} bm =  0:}
\CommentTok{\#\textgreater{}  trt\_indicator emmean   SE  df lower.CL upper.CL}
\CommentTok{\#\textgreater{}              0   21.0 1.17 141     18.7     23.3}
\CommentTok{\#\textgreater{}              1   17.7 1.17 141     15.4     20.0}
\CommentTok{\#\textgreater{} }
\CommentTok{\#\textgreater{} bm =  1:}
\CommentTok{\#\textgreater{}  trt\_indicator emmean   SE  df lower.CL upper.CL}
\CommentTok{\#\textgreater{}              0   23.5 1.85 141     19.8     27.2}
\CommentTok{\#\textgreater{}              1   16.4 1.85 141     12.8     20.1}
\CommentTok{\#\textgreater{} }
\CommentTok{\#\textgreater{} Results are averaged over the levels of: period }
\CommentTok{\#\textgreater{} Degrees{-}of{-}freedom method: kenward{-}roger }
\CommentTok{\#\textgreater{} Confidence level used: 0.95}
\end{Highlighting}
\end{Shaded}

\textbf{Interpretation}: Can estimate population-level effect
modification, but cannot estimate individual-specific interaction terms.

\subsection{Simulate an Aggregated N-of-1 Dataset with Biomarker and
Carryover}\label{simulate-an-aggregated-n-of-1-dataset-with-biomarker-and-carryover}

\begin{Shaded}
\begin{Highlighting}[]
\FunctionTok{set.seed}\NormalTok{(}\DecValTok{2025}\NormalTok{)}

\NormalTok{n\_subj }\OtherTok{\textless{}{-}} \DecValTok{25}
\NormalTok{cycles }\OtherTok{\textless{}{-}} \DecValTok{6}
\NormalTok{obs\_per\_period }\OtherTok{\textless{}{-}} \DecValTok{7}

\CommentTok{\# True effects}
\NormalTok{mu }\OtherTok{\textless{}{-}} \DecValTok{20}
\NormalTok{beta\_treat\_pop }\OtherTok{\textless{}{-}} \SpecialCharTok{{-}}\DecValTok{4}
\NormalTok{beta\_bm }\OtherTok{\textless{}{-}} \DecValTok{2}
\NormalTok{beta\_int\_pop }\OtherTok{\textless{}{-}} \SpecialCharTok{{-}}\FloatTok{1.5}
\NormalTok{beta\_carry }\OtherTok{\textless{}{-}} \FloatTok{1.5}

\NormalTok{sigma\_subj }\OtherTok{\textless{}{-}} \DecValTok{6}
\NormalTok{sigma\_treat\_sd }\OtherTok{\textless{}{-}} \DecValTok{3}
\NormalTok{sigma\_int\_sd }\OtherTok{\textless{}{-}} \FloatTok{0.8}
\NormalTok{sigma\_resid }\OtherTok{\textless{}{-}} \DecValTok{5}

\CommentTok{\# Create subject{-}level random effects}
\NormalTok{subj }\OtherTok{\textless{}{-}} \FunctionTok{tibble}\NormalTok{(}
  \AttributeTok{subject =} \DecValTok{1}\SpecialCharTok{:}\NormalTok{n\_subj,}
  \AttributeTok{u =} \FunctionTok{rnorm}\NormalTok{(n\_subj, }\DecValTok{0}\NormalTok{, sigma\_subj),}
  \AttributeTok{u\_trt =} \FunctionTok{rnorm}\NormalTok{(n\_subj, }\DecValTok{0}\NormalTok{, sigma\_treat\_sd),}
  \AttributeTok{u\_int =} \FunctionTok{rnorm}\NormalTok{(n\_subj, }\DecValTok{0}\NormalTok{, sigma\_int\_sd),}
  \AttributeTok{bm\_baseline =} \FunctionTok{rnorm}\NormalTok{(n\_subj, }\DecValTok{0}\NormalTok{, }\DecValTok{1}\NormalTok{)}
\NormalTok{)}

\CommentTok{\# Build periods alternating A/B starting at random}
\NormalTok{nof1 }\OtherTok{\textless{}{-}}\NormalTok{ subj }\SpecialCharTok{\%\textgreater{}\%}
  \FunctionTok{crossing}\NormalTok{(}\AttributeTok{period =} \DecValTok{1}\SpecialCharTok{:}\NormalTok{cycles) }\SpecialCharTok{\%\textgreater{}\%}
  \FunctionTok{group\_by}\NormalTok{(subject) }\SpecialCharTok{\%\textgreater{}\%}
  \FunctionTok{mutate}\NormalTok{(}\AttributeTok{order\_start =} \FunctionTok{first}\NormalTok{(}\FunctionTok{sample}\NormalTok{(}\FunctionTok{c}\NormalTok{(}\DecValTok{0}\NormalTok{, }\DecValTok{1}\NormalTok{), }\DecValTok{1}\NormalTok{))) }\SpecialCharTok{\%\textgreater{}\%}
  \FunctionTok{ungroup}\NormalTok{() }\SpecialCharTok{\%\textgreater{}\%}
  \FunctionTok{mutate}\NormalTok{(}\AttributeTok{trt\_indicator =}\NormalTok{ (order\_start }\SpecialCharTok{+}\NormalTok{ period) }\SpecialCharTok{\%\%} \DecValTok{2}\NormalTok{)}

\CommentTok{\# Expand to daily observations within each period}
\NormalTok{nof1 }\OtherTok{\textless{}{-}}\NormalTok{ nof1 }\SpecialCharTok{\%\textgreater{}\%}
  \FunctionTok{crossing}\NormalTok{(}\AttributeTok{day =} \DecValTok{1}\SpecialCharTok{:}\NormalTok{obs\_per\_period) }\SpecialCharTok{\%\textgreater{}\%}
  \FunctionTok{arrange}\NormalTok{(subject, period, day)}

\CommentTok{\# Add time{-}varying biomarker}
\NormalTok{nof1 }\OtherTok{\textless{}{-}}\NormalTok{ nof1 }\SpecialCharTok{\%\textgreater{}\%}
  \FunctionTok{mutate}\NormalTok{(}\AttributeTok{bm =}\NormalTok{ bm\_baseline }\SpecialCharTok{+} \FunctionTok{rnorm}\NormalTok{(}\FunctionTok{n}\NormalTok{(), }\DecValTok{0}\NormalTok{, }\FloatTok{0.3}\NormalTok{))}

\CommentTok{\# Add carryover effect}
\NormalTok{nof1 }\OtherTok{\textless{}{-}}\NormalTok{ nof1 }\SpecialCharTok{\%\textgreater{}\%}
  \FunctionTok{group\_by}\NormalTok{(subject) }\SpecialCharTok{\%\textgreater{}\%}
  \FunctionTok{mutate}\NormalTok{(}
    \AttributeTok{prior\_trt =} \FunctionTok{lag}\NormalTok{(trt\_indicator, }\AttributeTok{default =} \DecValTok{0}\NormalTok{),}
    \AttributeTok{carryover =} \FunctionTok{if\_else}\NormalTok{(period }\SpecialCharTok{\textgreater{}} \DecValTok{1} \SpecialCharTok{\&}\NormalTok{ day }\SpecialCharTok{\textless{}=} \DecValTok{2}\NormalTok{, prior\_trt, }\DecValTok{0}\NormalTok{)}
\NormalTok{  ) }\SpecialCharTok{\%\textgreater{}\%}
  \FunctionTok{ungroup}\NormalTok{()}

\CommentTok{\# Simulate outcomes}
\NormalTok{nof1 }\OtherTok{\textless{}{-}}\NormalTok{ nof1 }\SpecialCharTok{\%\textgreater{}\%}
  \FunctionTok{mutate}\NormalTok{(}
    \AttributeTok{Y =}\NormalTok{ mu }\SpecialCharTok{+}\NormalTok{ u }\SpecialCharTok{+}
\NormalTok{      (beta\_treat\_pop }\SpecialCharTok{+}\NormalTok{ u\_trt) }\SpecialCharTok{*}\NormalTok{ trt\_indicator }\SpecialCharTok{+}
\NormalTok{      beta\_bm }\SpecialCharTok{*}\NormalTok{ bm }\SpecialCharTok{+}
\NormalTok{      (beta\_int\_pop }\SpecialCharTok{+}\NormalTok{ u\_int) }\SpecialCharTok{*}\NormalTok{ trt\_indicator }\SpecialCharTok{*}\NormalTok{ bm }\SpecialCharTok{+}
\NormalTok{      beta\_carry }\SpecialCharTok{*}\NormalTok{ carryover }\SpecialCharTok{+}
      \FunctionTok{rnorm}\NormalTok{(}\FunctionTok{n}\NormalTok{(), }\DecValTok{0}\NormalTok{, sigma\_resid)}
\NormalTok{  )}

\CommentTok{\# Inspect}
\NormalTok{nof1 }\SpecialCharTok{\%\textgreater{}\%}
  \FunctionTok{group\_by}\NormalTok{(subject) }\SpecialCharTok{\%\textgreater{}\%}
  \FunctionTok{summarise}\NormalTok{(}\AttributeTok{n\_obs =} \FunctionTok{n}\NormalTok{(), }\AttributeTok{n\_periods =} \FunctionTok{n\_distinct}\NormalTok{(period)) }\SpecialCharTok{\%\textgreater{}\%}
  \FunctionTok{head}\NormalTok{()}
\CommentTok{\#\textgreater{} \# A tibble: 6 x 3}
\CommentTok{\#\textgreater{}   subject n\_obs n\_periods}
\CommentTok{\#\textgreater{}     \textless{}int\textgreater{} \textless{}int\textgreater{}     \textless{}int\textgreater{}}
\CommentTok{\#\textgreater{} 1       1    42         6}
\CommentTok{\#\textgreater{} 2       2    42         6}
\CommentTok{\#\textgreater{} 3       3    42         6}
\CommentTok{\#\textgreater{} 4       4    42         6}
\CommentTok{\#\textgreater{} 5       5    42         6}
\CommentTok{\#\textgreater{} 6       6    42         6}
\end{Highlighting}
\end{Shaded}

\subsection{Fit Aggregated N-of-1 Hierarchical Mixed Model with
Biomarker
Interaction}\label{fit-aggregated-n-of-1-hierarchical-mixed-model-with-biomarker-interaction}

\begin{Shaded}
\begin{Highlighting}[]
\CommentTok{\# Random intercept + random slope for treatment + biomarker interaction + carryover}
\NormalTok{m\_nof1 }\OtherTok{\textless{}{-}} \FunctionTok{lmer}\NormalTok{(}
\NormalTok{  Y }\SpecialCharTok{\textasciitilde{}}\NormalTok{ trt\_indicator }\SpecialCharTok{*}\NormalTok{ bm }\SpecialCharTok{+}\NormalTok{ carryover }\SpecialCharTok{+}\NormalTok{ (}\DecValTok{1} \SpecialCharTok{+}\NormalTok{ trt\_indicator }\SpecialCharTok{|}\NormalTok{ subject),}
  \AttributeTok{data =}\NormalTok{ nof1,}
  \AttributeTok{REML =} \ConstantTok{TRUE}
\NormalTok{)}
\FunctionTok{summary}\NormalTok{(m\_nof1)}
\CommentTok{\#\textgreater{} Linear mixed model fit by REML [\textquotesingle{}lmerMod\textquotesingle{}]}
\CommentTok{\#\textgreater{} Formula: Y \textasciitilde{} trt\_indicator * bm + carryover + (1 + trt\_indicator | subject)}
\CommentTok{\#\textgreater{}    Data: nof1}
\CommentTok{\#\textgreater{} }
\CommentTok{\#\textgreater{} REML criterion at convergence: 6448.2}
\CommentTok{\#\textgreater{} }
\CommentTok{\#\textgreater{} Scaled residuals: }
\CommentTok{\#\textgreater{}     Min      1Q  Median      3Q     Max }
\CommentTok{\#\textgreater{} {-}3.2778 {-}0.6249 {-}0.0146  0.6205  3.3358 }
\CommentTok{\#\textgreater{} }
\CommentTok{\#\textgreater{} Random effects:}
\CommentTok{\#\textgreater{}  Groups   Name          Variance Std.Dev. Corr}
\CommentTok{\#\textgreater{}  subject  (Intercept)   33.09    5.753        }
\CommentTok{\#\textgreater{}           trt\_indicator 10.86    3.295    0.02}
\CommentTok{\#\textgreater{}  Residual               23.89    4.887        }
\CommentTok{\#\textgreater{} Number of obs: 1050, groups:  subject, 25}
\CommentTok{\#\textgreater{} }
\CommentTok{\#\textgreater{} Fixed effects:}
\CommentTok{\#\textgreater{}                  Estimate Std. Error t value}
\CommentTok{\#\textgreater{} (Intercept)       21.8463     1.1736  18.614}
\CommentTok{\#\textgreater{} trt\_indicator     {-}3.9532     0.7293  {-}5.420}
\CommentTok{\#\textgreater{} bm                 1.5937     0.5273   3.023}
\CommentTok{\#\textgreater{} carryover          0.7476     0.4672   1.600}
\CommentTok{\#\textgreater{} trt\_indicator:bm  {-}1.2804     0.5776  {-}2.217}
\CommentTok{\#\textgreater{} }
\CommentTok{\#\textgreater{} Correlation of Fixed Effects:}
\CommentTok{\#\textgreater{}             (Intr) trt\_nd bm     crryvr}
\CommentTok{\#\textgreater{} trt\_indictr {-}0.039                     }
\CommentTok{\#\textgreater{} bm           0.061 {-}0.042              }
\CommentTok{\#\textgreater{} carryover   {-}0.050  0.001 {-}0.046       }
\CommentTok{\#\textgreater{} trt\_ndctr:b {-}0.028  0.110 {-}0.463  0.020}

\CommentTok{\# Extract fixed (population) effects}
\FunctionTok{fixef}\NormalTok{(m\_nof1)}
\CommentTok{\#\textgreater{}      (Intercept)    trt\_indicator               bm        carryover }
\CommentTok{\#\textgreater{}       21.8463025       {-}3.9531711        1.5937459        0.7476118 }
\CommentTok{\#\textgreater{} trt\_indicator:bm }
\CommentTok{\#\textgreater{}       {-}1.2803543}

\CommentTok{\# Population treatment effect (at mean biomarker = 0)}
\FunctionTok{fixef}\NormalTok{(m\_nof1)[}\StringTok{"trt\_indicator"}\NormalTok{]}
\CommentTok{\#\textgreater{} trt\_indicator }
\CommentTok{\#\textgreater{}     {-}3.953171}

\CommentTok{\# Population biomarker x treatment interaction}
\FunctionTok{fixef}\NormalTok{(m\_nof1)[}\StringTok{"trt\_indicator:bm"}\NormalTok{]}
\CommentTok{\#\textgreater{} trt\_indicator:bm }
\CommentTok{\#\textgreater{}        {-}1.280354}

\CommentTok{\# Carryover effect}
\FunctionTok{fixef}\NormalTok{(m\_nof1)[}\StringTok{"carryover"}\NormalTok{]}
\CommentTok{\#\textgreater{} carryover }
\CommentTok{\#\textgreater{} 0.7476118}

\CommentTok{\# Extract subject{-}specific treatment effects (BLUPs)}
\NormalTok{subj\_trt\_effects }\OtherTok{\textless{}{-}} \FunctionTok{ranef}\NormalTok{(m\_nof1)}\SpecialCharTok{$}\NormalTok{subject[[}\StringTok{"trt\_indicator"}\NormalTok{]] }\SpecialCharTok{+}
  \FunctionTok{fixef}\NormalTok{(m\_nof1)[}\StringTok{"trt\_indicator"}\NormalTok{]}
\FunctionTok{head}\NormalTok{(subj\_trt\_effects)}
\CommentTok{\#\textgreater{} [1] {-}6.690817 {-}7.458577 {-}3.954958 {-}5.074274 {-}1.525257 {-}9.434112}
\end{Highlighting}
\end{Shaded}

\subsubsection{Advanced Model with Random Biomarker
Interaction}\label{advanced-model-with-random-biomarker-interaction}

\begin{Shaded}
\begin{Highlighting}[]
\CommentTok{\# This estimates individual{-}specific biomarker x treatment interactions}
\CommentTok{\# May need more data or regularization for convergence}
\NormalTok{m\_nof1\_full }\OtherTok{\textless{}{-}} \FunctionTok{lmer}\NormalTok{(}
\NormalTok{  Y }\SpecialCharTok{\textasciitilde{}}\NormalTok{ trt\_indicator }\SpecialCharTok{*}\NormalTok{ bm }\SpecialCharTok{+}\NormalTok{ carryover }\SpecialCharTok{+}
\NormalTok{    (}\DecValTok{1} \SpecialCharTok{+}\NormalTok{ trt\_indicator }\SpecialCharTok{+}\NormalTok{ trt\_indicator}\SpecialCharTok{:}\NormalTok{bm }\SpecialCharTok{|}\NormalTok{ subject),}
  \AttributeTok{data =}\NormalTok{ nof1,}
  \AttributeTok{REML =} \ConstantTok{TRUE}\NormalTok{,}
  \AttributeTok{control =} \FunctionTok{lmerControl}\NormalTok{(}\AttributeTok{optimizer =} \StringTok{"bobyqa"}\NormalTok{)}
\NormalTok{)}
\FunctionTok{summary}\NormalTok{(m\_nof1\_full)}
\end{Highlighting}
\end{Shaded}

\subsection{Diagnostics and
Visualization}\label{diagnostics-and-visualization}

\begin{Shaded}
\begin{Highlighting}[]
\CommentTok{\# Plot individual subject treatment effects}
\NormalTok{subj\_effects }\OtherTok{\textless{}{-}} \FunctionTok{ranef}\NormalTok{(m\_nof1)}\SpecialCharTok{$}\NormalTok{subject }\SpecialCharTok{\%\textgreater{}\%}
  \FunctionTok{as.data.frame}\NormalTok{() }\SpecialCharTok{\%\textgreater{}\%}
  \FunctionTok{rownames\_to\_column}\NormalTok{(}\AttributeTok{var =} \StringTok{"subject"}\NormalTok{) }\SpecialCharTok{\%\textgreater{}\%}
  \FunctionTok{mutate}\NormalTok{(}
    \AttributeTok{subject =} \FunctionTok{as.integer}\NormalTok{(subject),}
    \AttributeTok{subj\_trt =}\NormalTok{ trt\_indicator }\SpecialCharTok{+} \FunctionTok{fixef}\NormalTok{(m\_nof1)[}\StringTok{"trt\_indicator"}\NormalTok{]}
\NormalTok{  )}

\FunctionTok{ggplot}\NormalTok{(subj\_effects, }\FunctionTok{aes}\NormalTok{(}\AttributeTok{x =}\NormalTok{ subj\_trt)) }\SpecialCharTok{+}
  \FunctionTok{geom\_histogram}\NormalTok{(}\AttributeTok{binwidth =} \FloatTok{0.5}\NormalTok{, }\AttributeTok{fill =} \StringTok{"steelblue"}\NormalTok{, }\AttributeTok{color =} \StringTok{"white"}\NormalTok{) }\SpecialCharTok{+}
  \FunctionTok{geom\_vline}\NormalTok{(}
    \AttributeTok{xintercept =} \FunctionTok{fixef}\NormalTok{(m\_nof1)[}\StringTok{"trt\_indicator"}\NormalTok{],}
    \AttributeTok{linetype =} \StringTok{"dashed"}\NormalTok{,}
    \AttributeTok{color =} \StringTok{"red"}
\NormalTok{  ) }\SpecialCharTok{+}
  \FunctionTok{labs}\NormalTok{(}
    \AttributeTok{title =} \StringTok{"Distribution of Individual Treatment Effects (BLUPs)"}\NormalTok{,}
    \AttributeTok{subtitle =} \StringTok{"Red line = population average effect"}\NormalTok{,}
    \AttributeTok{x =} \StringTok{"Individual treatment effect (CAPS change)"}\NormalTok{,}
    \AttributeTok{y =} \StringTok{"Count"}
\NormalTok{  ) }\SpecialCharTok{+}
  \FunctionTok{theme\_minimal}\NormalTok{()}
\end{Highlighting}
\end{Shaded}

\begin{figure}
\centering
\pandocbounded{\includegraphics[keepaspectratio,alt={Distribution of individual treatment effects}]{chat_files/figure-latex/viz-individual-effects-1.pdf}}
\caption{Distribution of individual treatment effects}
\end{figure}

\begin{Shaded}
\begin{Highlighting}[]
\CommentTok{\# Plot biomarker x treatment interaction}
\NormalTok{bm\_range }\OtherTok{\textless{}{-}} \FunctionTok{seq}\NormalTok{(}\SpecialCharTok{{-}}\DecValTok{2}\NormalTok{, }\DecValTok{2}\NormalTok{, }\AttributeTok{by =} \FloatTok{0.1}\NormalTok{)}
\NormalTok{interaction\_df }\OtherTok{\textless{}{-}} \FunctionTok{tibble}\NormalTok{(}
  \AttributeTok{bm =}\NormalTok{ bm\_range,}
  \AttributeTok{trt\_effect =} \FunctionTok{fixef}\NormalTok{(m\_nof1)[}\StringTok{"trt\_indicator"}\NormalTok{] }\SpecialCharTok{+}
    \FunctionTok{fixef}\NormalTok{(m\_nof1)[}\StringTok{"trt\_indicator:bm"}\NormalTok{] }\SpecialCharTok{*}\NormalTok{ bm\_range}
\NormalTok{)}

\FunctionTok{ggplot}\NormalTok{(interaction\_df, }\FunctionTok{aes}\NormalTok{(}\AttributeTok{x =}\NormalTok{ bm, }\AttributeTok{y =}\NormalTok{ trt\_effect)) }\SpecialCharTok{+}
  \FunctionTok{geom\_line}\NormalTok{(}\AttributeTok{linewidth =} \FloatTok{1.2}\NormalTok{, }\AttributeTok{color =} \StringTok{"darkgreen"}\NormalTok{) }\SpecialCharTok{+}
  \FunctionTok{geom\_hline}\NormalTok{(}\AttributeTok{yintercept =} \DecValTok{0}\NormalTok{, }\AttributeTok{linetype =} \StringTok{"dotted"}\NormalTok{) }\SpecialCharTok{+}
  \FunctionTok{labs}\NormalTok{(}
    \AttributeTok{title =} \StringTok{"Biomarker × Treatment Interaction"}\NormalTok{,}
    \AttributeTok{subtitle =} \StringTok{"Treatment effect as function of biomarker level"}\NormalTok{,}
    \AttributeTok{x =} \StringTok{"Biomarker (standardized)"}\NormalTok{,}
    \AttributeTok{y =} \StringTok{"Treatment effect (CAPS change)"}
\NormalTok{  ) }\SpecialCharTok{+}
  \FunctionTok{theme\_minimal}\NormalTok{()}
\end{Highlighting}
\end{Shaded}

\begin{figure}
\centering
\pandocbounded{\includegraphics[keepaspectratio,alt={Biomarker by treatment interaction}]{chat_files/figure-latex/viz-interaction-1.pdf}}
\caption{Biomarker by treatment interaction}
\end{figure}

\begin{Shaded}
\begin{Highlighting}[]
\CommentTok{\# Observed trajectories for sample subjects}
\NormalTok{sample\_subj }\OtherTok{\textless{}{-}} \FunctionTok{sample}\NormalTok{(}\FunctionTok{unique}\NormalTok{(nof1}\SpecialCharTok{$}\NormalTok{subject), }\DecValTok{6}\NormalTok{)}

\NormalTok{nof1 }\SpecialCharTok{\%\textgreater{}\%}
  \FunctionTok{filter}\NormalTok{(subject }\SpecialCharTok{\%in\%}\NormalTok{ sample\_subj) }\SpecialCharTok{\%\textgreater{}\%}
  \FunctionTok{mutate}\NormalTok{(}
    \AttributeTok{time =}\NormalTok{ (period }\SpecialCharTok{{-}} \DecValTok{1}\NormalTok{) }\SpecialCharTok{*}\NormalTok{ obs\_per\_period }\SpecialCharTok{+}\NormalTok{ day,}
    \AttributeTok{trt =} \FunctionTok{if\_else}\NormalTok{(trt\_indicator }\SpecialCharTok{==} \DecValTok{1}\NormalTok{, }\StringTok{"Active"}\NormalTok{, }\StringTok{"Placebo"}\NormalTok{)}
\NormalTok{  ) }\SpecialCharTok{\%\textgreater{}\%}
  \FunctionTok{ggplot}\NormalTok{(}\FunctionTok{aes}\NormalTok{(}\AttributeTok{x =}\NormalTok{ time, }\AttributeTok{y =}\NormalTok{ Y, }\AttributeTok{color =}\NormalTok{ trt)) }\SpecialCharTok{+}
  \FunctionTok{geom\_line}\NormalTok{(}\AttributeTok{alpha =} \FloatTok{0.7}\NormalTok{) }\SpecialCharTok{+}
  \FunctionTok{geom\_point}\NormalTok{(}\AttributeTok{size =} \FloatTok{0.8}\NormalTok{) }\SpecialCharTok{+}
  \FunctionTok{facet\_wrap}\NormalTok{(}\SpecialCharTok{\textasciitilde{}}\NormalTok{subject, }\AttributeTok{scales =} \StringTok{"free\_y"}\NormalTok{) }\SpecialCharTok{+}
  \FunctionTok{labs}\NormalTok{(}
    \AttributeTok{title =} \StringTok{"Individual Patient Trajectories"}\NormalTok{,}
    \AttributeTok{x =} \StringTok{"Day"}\NormalTok{,}
    \AttributeTok{y =} \StringTok{"CAPS score"}\NormalTok{,}
    \AttributeTok{color =} \StringTok{"Treatment"}
\NormalTok{  ) }\SpecialCharTok{+}
  \FunctionTok{theme\_minimal}\NormalTok{()}
\end{Highlighting}
\end{Shaded}

\begin{figure}
\centering
\pandocbounded{\includegraphics[keepaspectratio,alt={Individual patient trajectories}]{chat_files/figure-latex/viz-trajectories-1.pdf}}
\caption{Individual patient trajectories}
\end{figure}

\subsection{Notes on Extensions and Real-Data
Considerations}\label{notes-on-extensions-and-real-data-considerations}

\begin{itemize}
\item
  For real CAPS outcomes measured daily, expand the simulation to
  include within-period autocorrelation (e.g., AR(1)), measurement
  error, and missingness. Use \texttt{nlme::lme} or specify correlation
  structures for autocorrelation.
\item
  If carryover is suspected, consider adding period-by-treatment
  interaction terms or explicitly modeling carryover terms; tests for
  carryover should be planned but interpreted cautiously (Jones \&
  Kenward, 2014).
\item
  In the aggregated N-of-1 context, Bayesian hierarchical models (e.g.,
  \texttt{brms}) allow flexible priors and full posterior inference for
  individual effects (Zucker et al., 2010).
\item
  For regulatory-facing analyses, prespecify the primary estimand
  (population mean vs individual responder analysis), multiplicity
  handling, and sensitivity analyses for missing data and carryover.
\end{itemize}

\end{document}
